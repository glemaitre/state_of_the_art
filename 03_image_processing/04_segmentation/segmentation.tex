\subsection{Segmentation} \label{subsec:segmentation}

Segmentation task consists in delineating the prostate boundaries in the \ac{mri}. This procedure occurs to be predominant by focusing only on the organ of interest for the further fusions of the multi-modalities set of data (\cite{Ghose2012}). In this section, only the segmentation methods used in \ac{cad} systems will presented. An entire review of prostate segmentation methods in \ac{mri} can be found in \cite{Ghose2012}.

\begin{table}
	\caption{Overview of the segmentation methods used in \ac{cad} systems.}
	\small
	%\renewcommand{\arraystretch}{1.5}
	\begin{tabular}{p{.65\linewidth} p{.25\linewidth}}
		\hline \\ [-1.5ex]
		\textbf{Segmentation methods} & \textbf{References} \\ \\ [-1.5ex]
		\hline \\ [-1.5ex]
		\textit{\ac{mri}-based segmentation:} & \\ \\ [-1.5ex]
		\quad Manual segmentation & $[$4-5,16,18-21,24,38-40$]$ \\
		\quad Region-based segmentation & $[$11$]$ \\
		\quad Hybrid segmentation & $[$10,34-36,41$]$ \\ \\ [-1.5ex]
		\textit{\ac{mrsi}-based segmentation:} & \\ \\ [-1.5ex]
		\quad Clustering & $[$28$]$ \\ \\ [-1.5ex]
		\hline
	\end{tabular}
\end{table}

\subsubsection{MRI-based segmentation}

\setenumerate{listparindent=\parindent,itemsep=10px}
\setlist{noitemsep}
\begin{enumerate}[leftmargin=*]

\item[$-$] \textbf{\textit{Manual segmentation:}} {\color{red} TO CHECK AGAIN, maybe forgot some CAD for this item.} To highlight the importance of prostate segmentation task in \ac{cad} systems, it is interesting to note the large number of studies which at least segment manually the prostate organs (\cite{Artan2009,Artan2010,Matulewicz2013,Niaf2011,Niaf2012,Ozer2009,Ozer2010,Puech2009,Vos2008,Vos2008a}).

\item[$-$] \textbf{\textit{Region-based segmentation:}}

\cite{Litjens2012} used a multi-atlas-based segmentation using multi-modal images (e.g., \ac{t2w}-\ac{mri} and \ac{adc} map) to segment the prostate with an additional pattern recognition method to differentiate \ac{cg} and \ac{pz} as proposed in \cite{Litjens2012a}. This method consists in three different steps: (i) the registration between each atlas and the multi-modal images, (ii) the atlas selection and finally (iii) the classification of the prostate segmented voxels in either \ac{cg} or \ac{pz}. 

The registration between each atlas and the \ac{mri} images is performed using two successive registrations; the first registration is a rigid registration to roughly aligned the atlases and the \ac{mri} images and the second is an elastic registration using B-spline transformation. The objective function to perform the registration is defined as the weighted sum of the metric of both \ac{t2w}-\ac{mri} and \ac{adc} map. The metric is based on \ac{mi}. We refer to the next section for more details in regard to registration. Two strategies of atlas selection were performed by using either a majority voting approach or the \ac{staple} approach (\cite{Warfield2004}).

After these both stages, the prostate have been segmented and the differentiation of both \ac{cg} and \ac{pz} has to be carried out. This problem was carried out by classifying each voxel using \ac{lda}. Three types of features were considered: (i) anatomy, (ii) intensity and (iii) texture. Regarding the anatomy, relative position and relative distance from the pixel to the border of the prostate were used. The intensity features consist in the intensity of the voxel in the ADC coefficient and the T$_2$ map. The texture features were composed of five different features: homogeneity, correlation (\cite{Amadasun1989}), entropy, texture strength (\cite{Li2005a}) and \ac{lbp} (\cite{Ojala1996}). Finally, some morphological operations were applied to remove artefact and the contour between the zones were smooth using the \ac{tps} (\cite{Bookstein1989}).

\item[$-$] \textbf{\textit{Hybrid-based segmentation:}} 

\cite{Viswanath2008a,Viswanath2009} used a \ac{mantra} method as proposed by \cite{Toth2008}. \ac{mantra} is closely related to the \ac{asm} from \cite{Cootes1995}. This algorithm consists of two stages: (i) a training stage where a shape and appearance model is generated and (ii) the actual segmentation performed based on the learned model. 

For the training stage, a set of landmarks is defined and the shape model is generated as in the original \ac{asm} method (\cite{Cootes1995}). Then, to model the appearance, a set of $K$ texture images $\{I_1,I_2,\cdots,I_k\}$ based on first and second order statistical texture features, are computed. For a given landmark $l$ with its given neighbourhood $\mathcal{N}(l)$, its feature matrix extracted can be expressed as:

\begin{equation}
	f_l = \{ I_1(\mathcal{N}(l)), I_2(\mathcal{N}(l)), \cdots, I_k(\mathcal{N}(l)) \} \ .	
	\label{eq:mantra1}
\end{equation}

\noindent where $I_k(\mathcal{N}(l))$ represent a feature vector obtained by sampling the $k^{\text{th}}$ texture map using the neighbourhood $\mathcal{N}(l)$.

By generating multiple landmark in the same fashion than with the \ac{asm}, \ac{pca} (\cite{Pearson1901}) is applied to learned the appearance variations. 

For the segmentation stage, the mean shape learned previously is initialise in the test image. The same associated texture images as in the training stage are computed. For each landmark $l$, a neighbourhood of a larger size than $\mathcal{N}(l)$ is defined where multiple neighbourhoods $\mathcal{N}_R(l)$ of same size than $\mathcal{N}(l)$ will be generated. Each $\mathcal{N}_R(l)$ patch will be used to sample the texture images and a reconstruction will be performed using the appearance trained model. The new landmark location will be defined as the position where the \ac{mi} is maximal between the reconstructed and original values. This scheme is performed in a multi-resolution manner as in \cite{Cootes1995}.

\cite{Viswanath2012} used a \ac{weritas} proposed by \cite{Toth2009}. As \ac{mantra} method, this is really close to \ac{asm} formulation. 

In the training stage, a set of landmarks $\{l_1,l_2,\cdots,l_n\}$ associated with their given neighbourhood $\{\mathcal{N}(l_1),\mathcal{N}(l_2),\cdots,\mathcal{N}(l_n)\}$ are defined. 

In the same way, a set of pixels $\{p_1,p_2,\cdots,p_n\}$ associated with their given neighbourhood $\{\mathcal{N}(p_1),\mathcal{N}(p_2),\cdots,\mathcal{N}(p_n)\}$ are defined. 

A set of $K$ texture images $\{I_1,I_2,\cdots,I_k\}$ based on first and second order statistical texture features, are computed from the original image $I_0$. Hence, for a given landmark $l$ with its given neighbourhood $\mathcal{N}(l)$, its feature matrix extracted can be expressed as:

\begin{equation}
	f_l = \{ I_1(\mathcal{N}(l)), I_2(\mathcal{N}(l)), \cdots, I_k(\mathcal{N}(l)) \} \ .	
	\label{eq:weritas1}
\end{equation}

\noindent where $I_k(\mathcal{N}(l))$ represents a feature vector obtained by sampling the $k^{\text{th}}$ texture map using the neighbourhood $\mathcal{N}(l)$.

In the same way, for a given pixel $p$ with its given neighbourhood $\mathcal{N}(p)$, its feature matrix extracted can be expressed as:

\begin{equation}
	f_p = \{ I_1(\mathcal{N}(p)), I_2(\mathcal{N}(p)), \cdots, I_k(\mathcal{N}(p)) \} \ .	
	\label{eq:weritas2}
\end{equation}

\noindent where $I_k(\mathcal{N}(p))$ represents a feature vector obtained by sampling the $k^{\text{th}}$ texture map using the neighbourhood $\mathcal{N}(p)$.

Also, each of the feature $f_l$ and $f_p$ are associated with their means $\bar{f}_l$ and $\bar{f}_p$ and their covariance $\Phi_l$ and $\Phi_p$. A set of of Euclidean distance is computed for each landmark and pixel such that:

\begin{equation}
	E = \{ \| l - p \|_2 \} \ .
	\label{eq:weritas3}
\end{equation}

A set of Mahalanobis distance will also be computed such that:

blablablbalalaba

\cite{Litjens2011} and \cite{Vos2012} used an approach proposed by \cite{Huisman2010} in which the bladder, prostate and rectum are segmented tackling the segmentation task as an optimization problem. First, a probabilistic model is first trained by embedding the three following aspects: (i) the shape by defining each organ as an ellipse, (ii) the position by defining the distance and the angle between each organ center and (iii) the appearance using the \acp{pdf} of \ac{si} of each organ. \cite{Litjens2011} used only \ac{adc} map to encode the appearance whereas \cite{Vos2012} used both \ac{adc} and T$_2$ maps. Then, during the optimization using a quasi-Newton optimizer, an objective function is minimized. This function is defined as the sum of the deviations from the above model learnt. This rough segmentation is then used inside a Bayesian framework to refine the shape. {\color{red}Really have to check because things seem to be made up.}

\end{enumerate}

\subsubsection{MRSI-based segmentation}

\cite{Tiwari2009} localize the voxels corresponding to the prostate organ using a hierarchical spectral clustering. First, each \ac{mrsi} spectrum is projected into a lower dimension space using graph embedding \cite{Shi2000}. To proceed, a similarity matrix $W$ is computed using a Gaussian similarity measure from Euclidean distance (\cite{Belkin2001}) such that:

\begin{equation}
	W(\mathbf{x},\mathbf{y}) =
	\begin{cases}	
	 	\exp \left( \frac{\| s(\mathbf{x}) - s(\mathbf{y}) \|_2^2}{\sigma^2} \right) \ , & \text{if } \| \mathbf{x} - \mathbf{y} \|_2 < \epsilon \ , \\
	 	0 \ , & \text{if } \| \mathbf{x} - \mathbf{y} \|_2 > \epsilon \ .
	 \end{cases}
	\label{eq:ge1}
\end{equation}

\noindent where $s(\mathbf{x})$ and $s(\mathbf{y})$ are the \ac{mrsi} spectra for the voxels $\mathbf{x}$ and $\mathbf{y}$ respectively, $\sigma$ is the standard deviation of the Gaussian similarity measure and $\epsilon$ is the parameter to defined an $\epsilon$-neighbourhood.

The projection can be performed as a generalized eigenvector problem such that:
\begin{eqnarray}
	Lu & = & \lambda D u \ , \nonumber \\
	D(\mathbf{x},\mathbf{x}) & = & \sum_{\mathbf{y}} W(\mathbf{x},\mathbf{y}) \ , \label{eq:ge2} \\
	L & = & D-W \ . \nonumber
\end{eqnarray}

\noindent where $D$ is the diagonal weight matrix, $L$ is the Laplacian matrix, $\lambda$ and $u$ represent the eigenvalues and eigenvectors.

Once that the \ac{mrsi} spectra are projected into the lower dimension space, a replicate k-means clustering method is run defining two clusters. The larger cluster is assimilated to be the cluster corresponding to non-prostate voxels and will be eliminated. The full procedure is repeated until the total number of voxels left is inferior to a given number.