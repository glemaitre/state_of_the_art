\subsection{Registration} \label{subsec:registration}

The role of image registration is vital in \ac{cad} systems using multi-parametric \ac{mri} images. As it will be discussed in \ac{sec}\,\ref{sec:dataclassfra}, for the sake of an optimal classification, the features detected in each modality will be grouped depending of their spatial locations. Hence, one has to ensure the perfect alignment of the multi-modal \ac{mri} images ahead of performing any classification.

From \ac{sec}\,\ref{subsubsec:geotra} to \ac{sec}\,\ref{subsubsec:int}, we individually review the different components of a typical registration framework: transformation model, similarity metric, optimizer and interpolation. \Acl{sec}~\ref{subsubsec:regrev} will summarize the combinations of these components especially used in \ac{cad} systems. Exhaustive reviews covering registration methods in computer science and medical fields can be found in~\cite{Maintz1998,Zitova2003}, respectively.

\subsubsection{Geometric transformation models}\label{subsubsec:geotra}

From all \ac{cad} for \ac{cap} systems reviewed, only parametric transformation models have been used, mainly based on affine and elastic transformations. Affine transformations provide degrees of freedom managing rotations and translation as with the rigid transformations but also shearing and scaling. Elastic transformations offer the advantage to handle local distortions. Within the elastic transformations, two radial basis functions have been used in the reviewed \ac{cad} for \ac{cap} systems: (i) the \ac{tps} and (ii) the B-splines. Apart from the formalism, these two approaches have a main difference. With B-splines, the control points are usually uniformly and densely placed on a grid whereas with \ac{tps}, the control points correspond to detected or selected key points. For instance, by using \ac{tps}, Mitra et al. in~\cite{Mitra2011} obtained more accurate and time efficient results than with the B-splines strategy~\cite{Mitra2012a}.

In general only rigid or affine registrations have been used to register multi-parametric images from a same protocol whereas elastic registration methods are more commonly used to register multi-protocol images (e.g., histopathology with \ac{mri} images)~\cite{Toth2008,Toth2009}.

\subsubsection{Similarity measure}\label{subsubsec:simmea}

The most naive similarity measure used in the reviewed registration frameworks is the \acf{mse} of the \ac{si} of \ac{mri} images. However, this metric is not well suited when multi-parametric images are involved due to the tissue appearance variations between the different modalities.
In that regard, \ac{mi} was introduced as a registration measure in the late 1990's in~\cite{Pluim2003}. The \ac{mi} measure finds its foundation in the assumption that a homogeneous region in the first modality image should also appear as a homogeneous region in the second modality even if their \acp{si} are not identical. Thus, those regions share information and the registration task can be achieved by maximizing this common information. Maximizing the \ac{mi} is in fact equivalent to minimizing the joint entropy which is a measure related with the degree of uncertainty or dispersion of the data in the joint histogram. A generalized form of \ac{mi}, \ac{cmi}, was proposed in~\cite{Chappelow2011}. \ac{cmi} encompasses interdependent information such as texture and gradient into the metric.

\subsubsection{Optimization methods}\label{subsubsec:optmea}

Registration is usually regarded as an optimization problem where the parameters of the geometric transformation model have to be inferred by minimizing the similarity measure. Iterative estimation methods are commonly used being the L-BFGS-B quasi-Newton method~\cite{Byrd1995} and gradient descent~\cite{Viola1997} the most common ones. During our review, we noticed that authors do not usually linger over optimizer choice.

\subsubsection{Interpolation}\label{subsubsec:int}

The registration procedure involves transforming an image, and pixels mapped to non-integer points must be approximated using interpolation methods. As for the optimization methods, we notice that little attention has been paid on the choice of those interpolations methods. However, commonly used methods are bilinear, nearest-neighbour, bi-cubic, spline and inverse-distance weighting method~\cite{Mitra2012}.

\subsubsection{Registration methods used in \ac{cad} system}\label{subsubsec:regrev}

Studies presenting a \ac{cad} pipeline incorporating an automatic registration procedure are summarized in \ac{tab}~\ref{tab:regtab}. 
Ampeliotis et al. in~\cite{Ampeliotis2007,Ampeliotis2008} did not use a general registration framework  to register 2D \ac{t2w} and \ac{dce} images. By using image symmetries and the \ac{mse} metric, they found the parameters of an affine transformation but without using a common objective function. They were finding independently and sequentially the scale factor, the rotation and finally the translation.

Giannini et al.~\cite{Giannini2013} used also a in-house registration for 2D \ac{t2w} and \ac{dw} images using an affine model. The bladder is first segmented in both modalities in order to obtain its contours and to focus the registration.
Giannini et al.~\cite{Giannini2013} and also Vos et al.~\cite{Vos2010} used the same framework which is based on finding an affine transformation to register the \ac{t2w} and \ac{dce} images using \ac{mi}~\cite{Rueckert1999}. Then, an elastic registration using B-spline takes place using the affine parameters to initialize the geometric model with the  same similarity measure. However, the approaches differ regarding the choice of the optimizer since a gradient descent is used in~\cite{Giannini2013} and the same optimization problem is tackled via a quasi-Newton method in~\cite{Vos2010}. Moreover, Giannini et al.~\cite{Giannini2013} performed a 2D registration whereas Vos et al.~\cite{Vos2010} registered 3D volumes.

Viswanath et al. in~\cite{Viswanath2008a,Viswanath2009} as well as Vos et al.~\cite{Vos2008} performed an affine registration using the \ac{mi} as similarity measure to correct the misalignment between \ac{t2w} and \ac{dce} images. The choice of the optimizer was not specified. Viswanath et al.~\cite{Viswanath2008a,Viswanath2009} focused on 2D registration while Vos et al.~\cite{Vos2008} performed 3D registration.

Finally, Viswanath et al. in~\cite{Viswanath2011} performed a 3D registration with the three modalities, \ac{t2w} and \ac{dce} and \ac{dw} \ac{mri}, by using an affine transformation model combined with the \ac{cmi} similarity measure as presented in~\cite{Chappelow2011}. Moreover, in this latter work, the authors employed a gradient descent approach to solve this problem but suggested Nelder-Mead simplex and quasi-Newton method as other solutions.

%%% Local Variables: 
%%% mode: latex
%%% TeX-master: "../../g_lemaitre_state_of_the_art"
%%% End: 