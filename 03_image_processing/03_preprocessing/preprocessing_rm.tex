\subsection{Pre-processing} \label{subsec:preprocessing}

\subsubsection{\ac{mri} images pre-processing}

Three different groups of pre-processing methods are commonly applied to images as initial stage in \ac{cad}.

\setenumerate{listparindent=\parindent,itemsep=10px}
\setlist{noitemsep}
\begin{enumerate}[leftmargin=*]

\item[$-$] \textbf{\textit{Noise filtering:}} The \ac{nmr} signal measured and recorded in the k-space during an \ac{mri} acquisition is affected by noise. This noise obeys a complex Gaussian white noise mainly due to thermal noises in the patient area~\cite{Nowak1999}. Furthermore, \ac{mri} images visualized by radiologists are in fact the magnitude images resulting from the complex Fourier transform of the k-space data. The complex Fourier transform, being a linear and orthogonal transform, does not affect the Gaussian noise characteristics~\cite{Nowak1999}. However, the function involved in the magnitude computation is a non-linear transform (i.e., the square root of the sum of squares of real and the imaginary parts), implying that the noise distribution is no longer Gaussian; it indeed follows a Rician distribution making the denoising task harder. Briefly, a Rician distribution can be characterized as follows: in low-\ac{si} region (low \ac{snr}), it can be approximated with a Rayleigh distribution while in high-\ac{si} region (high \ac{snr}), it is similar to a Gaussian distribution~\cite{Manjon2008}. Reviews of all denoising methods can be found in~\cite{Buades2005,Mohan2014}.

  Median filtering is the simplest approach used to address the denoising issue in \ac{mri} images~\cite{Ozer2009,Ozer2010}. However, from a theoretical point of view, this simple filtering method is not well formalized to address the noise distribution in \ac{mri} images.

  More complex approaches were proposed to overcome this problem. A common method used to denoise \ac{mri} images is based on wavelet-based filtering. Investigations focus on the strategies to perform the most adequate coefficient shrinkage method (e.g., using thresholding, singularity property or Bayesian framework)~\cite{Pizurica2002}.

  Ampeliotis et al. in~\cite{Ampeliotis2007,Ampeliotis2008} performed wavelet shrinkage to denoise magnitude \ac{mri} images (cf., \ac{t2w}-\ac{mri} and \ac{dce}-\ac{mri}) using thresholding techniques~\cite{Mallat2008}. However, since the wavelet transform is an orthogonal transform, the Rician distribution of the noise is preserved in the wavelet-domain. Hence, for low \ac{snr}, the wavelet and scaling coefficients still suffer from a bias due to this specific noise distribution~\cite{Nowak1999}. 

  Lopes et al. in~\cite{Lopes2011} used the filtering technique proposed by~\cite{Pizurica2003} to denoise \ac{t2w}-\ac{mri} which was based on joint detection and estimation theory~\cite{Middleton1968}.

\item[$-$] \textbf{\textit{Bias correction:}} Besides being corrupted by noise, \ac{mri} images are also affected by the inhomogeneity of the \ac{mri} field commonly referred to as bias field~\cite{Styner2000}. This bias field results in a smooth variation of the \ac{si} through the image. When an endorectal coil is used, an artefact resulting of an hyper-intense signal can be observed around the coil on the images. As a consequence, the \ac{si} of identical tissues varies depending on their spatial location in the image making further processes such as segmentation or registration harder~\cite{Jungke1987,Vovk2007}. A review of bias correction methods can be found in~\cite{Vovk2007}.

  Viswanath et al. in~\cite{Viswanath2009} performed bias correction on \ac{t2w}-\ac{mri} using a parametric Legendre polynomial model proposed in~\cite{Styner2000} and available in the \ac{itk} library\footnote{The \ac{itk} library is available at: \url{http://www.itk.org/}}.

  Lv et al. in~\cite{Lv2009} corrected the inhomogeneity in \ac{t2w}-\ac{mri} images by using the method proposed in~\cite{Madabhushi2006}. In this method, the \ac{mri} images are corrected iteratively by successively detecting the image foreground via \ac{gscale} and estimating a bias field function based on a second-order polynomial model.

\item[$-$] \textbf{\textit{\Ac{si} normalization/standardization:}} As discussed in the later section, segmentation or classification tasks are usually performed by first learning from a training set of patients. Hence, one can emphasize the desire to perform \ac{mri} examinations with a high repeatability or in other words, one would like to obtain similar \ac{mri} images (cf., similar \acp{si}) for patients of the same group (cf., healthy patients \textit{vs.} patients with \ac{cap}), for a similar sequence.

  However, it is a known fact that variability between patients occurs during the \ac{mri} examinations even using the same scanner, protocol or sequence parameters~\cite{Nyul1999}. Hence, the aim of normalization or standardization of the \ac{mri} data is to remove the variability between patients and enforce the repeatability of the \ac{mri} examinations. Approaches used to standardize \ac{mri} images can be either categorized as statistical-based standardization or organ \ac{si}-based standardization. 

  Artan et al.~\cite{Artan2009,Artan2010} as well as Ozer et al.~\cite{Ozer2009,Ozer2010} standardized \ac{t2w}, \ac{dce} and \ac{dw} \ac{mri} images by computing the \textit{standard score} (also called \textit{z-score}) of the pixels of the \ac{pz}. In a similar way, Liu et al.~\cite{Liu2013} normalized \ac{t2w}-\ac{mri} by making use of the median and interquartile range for all the pixels.

  Lv et al.~\cite{Lv2009} scaled the \ac{si} of \ac{t2w}-\ac{mri} images using the method proposed in~\cite{Nyul2000} based on \ac{pdf} matching. This approach is based on the assumption that \ac{mri} images from the same sequence should share the same \ac{pdf} appearance. Hence, one can approach this issue by transforming and matching the \acp{pdf} using some statistical landmarks such as median and different quantiles.

  Viswanath et al. in~\cite{Viswanath2009,Viswanath2011,Viswanath2012} used a variant of this previous approach presented in~\cite{Madabhushi2006a} aiming to standardize the \ac{t2w}-\ac{mri} images. Instead of computing the \ac{pdf} of an entire image, a pre-segmentation of the foreground is carried out via \ac{gscale}.

  The methods described above were statistical-based methods. However, the standardization problem can be tackled by normalizing the MRI images using the \ac{si} of some known organs present in these images. Niaf et al.~\cite{Niaf2011,Niaf2012} normalized \ac{t2w}-\ac{mri} images by dividing the original \ac{si} of the images by the mean \ac{si} of the bladder. Likewise, in~\cite{Niaf2011} standardized the \ac{t1w}-\ac{mri} images using the \ac{aif} as proposed in~\cite{Wiart2007}.

\end{enumerate}

\subsubsection{\ac{mrsi} spectra}

Presented in \ac{sec}\,\ref{subsubsec:mrimrsi}, \ac{mrsi} is a modality related to a one dimensional signal. Hence, specific pre-processing steps for this type of signals have been applied instead of standard signal processing methods.

\setenumerate{listparindent=\parindent,itemsep=10px}
\setlist{noitemsep}
\begin{enumerate}[leftmargin=*]

\item[$-$] \textbf{\textit{Phase correction:}} \ac{mrsi} data acquired suffer from zero-order and first-order phase misalignments~\cite{Chen2002,Osorio-Garcia2012}). Parfait et al.~\cite{Parfait2012} used a method proposed in~\cite{Chen2002} where the phase of \ac{mrsi} signal is corrected based on entropy minimization in the frequency domain.

\item[$-$] \textbf{\textit{Water and lipid residuals filtering:}} The water and lipid metabolites occur in much higher concentrations than the metabolites of interests (cf., choline, creatine and citrate)~\cite{Zhu2010,Osorio-Garcia2012}. Fortunately, specific \ac{mrsi} sequences were developed in order to suppress water and lipid metabolites using pre-saturation techniques~\cite{Zhu2010}. However, these techniques do not perfectly remove water and lipids peaks and some residuals are still present in the \ac{mrsi} spectra. Therefore, different post-processing methods have been proposed to enhance the quality of the \ac{mrsi} spectra by removing these residuals.
  
  Kelm et al.~\cite{Kelm2007} used the well known HSVD algorithm proposed in~\cite{Pijnappel1992} which models the \ac{mrsi} signal by a sum of exponentially damped sinusoids in the time domain.

\item[$-$] \textbf{\textit{Baseline correction:}} Sometimes, the problem discussed in the above section regarding the lipid molecules is not addressed simultaneously with water residuals suppression. Lipids and macromolecules are known to affect the baseline of the \ac{mrsi} spectra. They could cause errors during further fitting processes aiming to quantify the metabolites, especially regarding the citrate metabolite.
  
  Parfait et al.~\cite{Parfait2012} made the comparison of two different methods to detect the baseline and correct the \ac{mrsi} spectra which are based on ~\cite{Lieber2003,Devos2004} in which Liber et al.~\cite{Lieber2003} addressed the problem of baseline detection in the frequency domain by iteratively fitting a polynomial of low degree. Parfait et al.~\cite{Parfait2012} modified this algorithm by convolving a Gaussian kernel to smooth the \ac{mrsi} signal instead of fitting a polynomial function. Unlike in~\cite{Lieber2003}, Devos et al. in~\cite{Devos2004} proposed to correct the baseline in the time domain by multiplying the \ac{mrsi} signal by a decreasing exponential function. However, Parfait et al.~\cite{Parfait2012} concluded that the method proposed in~\cite{Lieber2003} outperformed the one in~\cite{Devos2004}. 

  In the contemporary work of Tiwari et al.~\cite{Tiwari2012}, the authors detected the baseline using a local non-linear fitting method avoiding regions with significant peaks which were detected using a experimentally parametrised signal-to-noise ratio (i.e. a value larger than 5 dB).

\item[$-$] \textbf{\textit{Frequency alignment:}} Due to variations of the experimental conditions, a frequency shift can be observed in the \ac{mrsi} spectra~\cite{Chen2002,Osorio-Garcia2012}. Tiwari et al.~\cite{Tiwari2012} corrected the frequency shift by first detecting known metabolite peaks such as choline, creatine and citrate. The frequency shift is corrected by minimizing the frequency error between the experimental and theoretical values of each of these peaks.

\item[$-$] \textbf{\textit{Normalization:}} Due to variations of the experimental conditions, the \ac{mrsi} signal may also vary between patients.
  
  Parfait et al.~\cite{Parfait2012} as in~\cite{Devos2004} compared two methods to normalize the \ac{mrsi} signal. In each method, the original \ac{mrsi} spectra is divided by a normalization factor, similar to the intensity normalization described earlier. The first approach to obtain the normalization factor is based on an estimation of the water concentration. It is required to have an additional \ac{mrsi} sequence where the water metabolites are unsuppressed. Using this sequence, an estimation of the water concentration can be performed using the previously reported HSVD algorithm.  The second approach to normalization is based on using the L$_2$ norm of the \ac{mrsi} spectra. It should be noted that both studies concluded that the L$_2$ normalization was more efficient in their framework~\cite{Parfait2012,Devos2004}.
  
\end{enumerate}

%%% Local Variables: 
%%% mode: latex
%%% TeX-master: "../../g_lemaitre_state_of_the_art"
%%% End: 