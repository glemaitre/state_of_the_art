\section{Introduction} \label{sec:introduction}

During the last century, physicists focused on constantly innovating in terms of imaging techniques assisting radiologists to overcome different tasks as cancer detection and diagnosis. However, human diagnosis still suffers from low repeatability, synonymous with erroneous detection or interpretations of abnormalities throughout clinical decisions (\cite{Giger2008,Hambrock2013}). These errors are driven by two majors causes (\cite{Giger2008}). On the one hand, observer limitations (e.g., constrained human visual perception, fatigue or distraction) are the principal human issues. On the other hand, the second reason is linked to the clinical cases themselves, for instance due to unbalanced data (number of healthy cases more abundant than malignant cases) or overlapping structures resulting from limitations of imaging techniques.

Computer vision has given rise to many promising solutions, but, instead of focusing on fully automatic computerized systems, researchers have aimed at providing computer image analysis techniques to aid radiologists in their clinical decisions (\cite{Giger2008}). In fact, these investigations brought about both concepts of \ac{cade} and \ac{cadx} grouped under the acronym \acs{cad}. Since those first steps, evidence has shown that \acs{cad} systems enhance the diagnosis performance of radiologists. \cite{Chan1999} reported a significant 4~\% improvement in breast cancer detection, in accordance with later studies (\cite{Dean2006}). Similar conclusions were drawn in the case of lung nodule detection (\cite{Li2004}), colon cancer (\cite{Petrick2008}) and \ac{cap} as well (\cite{Hambrock2013}). \cite{Chan1999} also hypothesized that \acs{cad} systems will be even more efficient assisting inexperienced radiologists than senior radiologists. That hypothesis was tested by \cite{Hambrock2013} and was confirmed in case of \ac{cap} detection. In this particular study, inexperienced radiologists obtained equivalent performance to senior radiologists, both with the help of a \acs{cad} system whereas the accuracy of their diagnosis was significantly poorer without this assistance.

In contradiction with the aforementioned statement, \ac{cap} detection using \acs{cad} is a young technology due to the fact that \ac{mri} is the keystone medical imaging technique (\cite{Hegde2013}). Four distinct \ac{mri} modalities are employed in \acs{cad} for \ac{cap} and were mainly developed after the mid-1990s: (i) \ac{t2w} \ac{mri} (\cite{Hricak1983}), (ii) \ac{dce} \ac{mri} (\cite{HuchBoni1995}), (iii) \ac{mrsi} (\cite{Kurhanewicz1996}) and (iv) \ac{dw} \ac{mri} (\cite{Scheidler1999}). It can be noted that these techniques came into existence relatively recently mainly due to technological progress. In addition, the increase of magnetic field strength and the development of endorectal coil, both improved image spatial resolution (\cite{Swanson2001}) needed to perform more accurate diagnosis. It is for this matter that development of \acs{cad} for \ac{cap} is still lagging behind the other fields stated above.

The first study using \ac{mri} as inputs of a \acs{cad} system was published eleven years ago by \cite{Chan2003}. Despite this, no less than fifty studies have been reviewed for this survey since that seminal work. To the best of our knowledge, there is no review in the literature regarding the advancement of \acs{cad} systems devoted specifically to \ac{cap} detection and diagnosis. Thus, our aim with this survey is threefold: (i) provide an overview of developed \acs{cad} systems for \ac{cap} detection and diagnosis based on \ac{mri} modalities (ii) assess the different work and (iii) point out avenues for future work.

We also would like to emphasize the fact that this study will review all the details of the computer vision aspects of the different studies. Thus, this survey is more intended for a medical imaging audience rather than a purely experimented clinical audience.

As discussed further in Sect. \ref{subsubsec:CAD}, we identified and characterized a common framework regarding the \acs{cad} systems. Stages involved in \acs{cad} work-flow can be categorized into three distinctive processes: (i) image regularization, (ii) \ac{cade}, (iii) \ac{cadx} (see Fig.~\ref{fig:wkfcad}). Image regularisation focuses on formatting the data while \ac{cade} and \ac{cadx} allow to detect possible lesions and distinguish malignant from non-malignant tumours, respectively.

This paper is organized as follows: Sect. \ref{sec:background} deals with general information about human prostate and background about \ac{cap}. Methods regarding \ac{cap} screening and imaging techniques used are also presented as well as an introduction to the \acs{cad} framework. Sections \ref{sec:imaprocfra} - \ref{sec:dataclassfra} review techniques used in different steps involved in a \acs{cad} work-flow which will be our main contribution. Image regularization framework including pre-processing (Sect. \ref{subsec:preprocessing}), segmentation (Sect. \ref{subsec:segmentation}) and registration (Sect. \ref{subsec:registration}) will be covered as well as \ac{cade} and \ac{cadx} strategies (Sect. \ref{sec:dataclassfra}) identified. Results and discussion are reported in Sect. \ref{sec:discussion} followed by a concluding section. Deriving from this discussion, we made available to the research community a public online dataset aiming at overcoming some drawbacks risen during the discussion.

%%% Local Variables: 
%%% mode: latex
%%% TeX-master: "../g_lemaitre_state_of_the_art"
%%% End: 