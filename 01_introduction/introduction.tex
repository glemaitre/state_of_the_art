\section{Introduction} \label{sec:introduction}

During the last century, physicists have focused on constantly innovating in terms of imaging techniques assisting radiologists to improve cancer detection and diagnosis. However, human diagnosis still suffers from low repeatability, synonymous with erroneous detection or interpretations of abnormalities throughout clinical decisions~\cite{Giger2008,Hambrock2013}. These errors are driven by two majors causes~\cite{Giger2008}: observer limitations (e.g., constrained human visual perception, fatigue or distraction) and the complexity of the clinical cases themselves, for instance due to unbalanced data (number of healthy cases more abundant than malignant cases) or overlapping structures.

Computer vision has given rise to many promising solutions, but, instead of focusing on fully automatic computerized systems, researchers have aimed at providing computer image analysis techniques to aid radiologists in their clinical decisions~\cite{Giger2008}. In fact, these investigations brought about both concepts of \ac{cade} and \ac{cadx} grouped under the acronym \acs{cad}. Since those first steps, evidence has shown that \acs{cad} systems enhance the diagnosis performance of radiologists. Chan et al. reported a significant 4~\% improvement in breast cancer detection~\cite{Chan1999}, which has been confirmed in later studies~\cite{Dean2006}. Similar conclusions were drawn in the case of lung nodule detection~\cite{Li2004}, colon cancer~\cite{Petrick2008} and \ac{cap} as well~\cite{Hambrock2013}. Chan et al. also hypothesized that \acs{cad} systems will be even more efficient assisting inexperienced radiologists than senior radiologists~\cite{Chan1999}. This hypothesis was tested by Hambrock et al.~\cite{Hambrock2013} and was confirmed in the case of \ac{cap} detection. In this particular study, inexperienced radiologists obtained equivalent performance to senior radiologists, both using \acs{cad} whereas the accuracy of their diagnosis was significantly poorer without \acs{cad}'s help.

In contradiction with the aforementioned statement, \acs{cad} for \ac{cap} is a young technology due to the fact that it is based on \ac{mri}~\cite{Hegde2013}. Four distinct \ac{mri} modalities are employed in \ac{cap} diagnosis which were mainly developed after the mid-1990s: (i) \ac{t2w} \ac{mri}~\cite{Hricak1983}, (ii) \ac{dce} \ac{mri}~\cite{HuchBoni1995}, (iii) \ac{mrsi}~\cite{Kurhanewicz1996} and (iv) \ac{dw} \ac{mri}~\cite{Scheidler1999}. In addition, the increase of magnetic field strength (from 1.5 to 3 Tesla) and the development of endorectal coils, both improved image spatial resolution~\cite{Swanson2001} needed to perform more accurate diagnosis. It is for this matter that the development of \acs{cad} for \ac{cap} is still lagging behind the other fields stated above.

The first study on \acs{cad} for \ac{mri} was published in 2003 by Chan et al.~\cite{Chan2003}. Despite this, no less than fifty studies have been reviewed for this survey since that seminal work. To the best of our knowledge, there is no fully detailed review in the literature regarding the advancement of \acs{cad} systems devoted specifically to \ac{cap} detection and diagnosis. Only the recent work of~\cite{Wang2014} briefly covers a reduced number of current research works on \acs{cad} for \ac{cap} (a total of $70$ references), but it does not provide a detailed description of any of the steps of a CAD system as it is done here with more than $200$ references. Moreover, a new categorisation is proposed in this work taking the clinical and technical aspects of a CAD system into account. Thus, the aim of this survey is threefold: (i) provide an overview and categorisation of the developed \acs{cad} systems for \ac{cap} detection and diagnosis based on \ac{mri} modalities (ii) assess the different works and (iii) point out avenues for future directions.

We also would like to emphasize the fact that this study will review the details of the computer vision aspects of the different studies. Thus, this survey is more intended for a medical imaging audience rather than a purely experimented clinical audience.

As discussed further in \acs{sec}\,\ref{subsubsec:CAD}, we identified and characterized a common framework regarding the \acs{cad} systems. Stages involved in \acs{cad} work-flow can be categorized into three distinctive processes: (i) image regularization, (ii) \ac{cade}, (iii) \ac{cadx} (see Fig.~\ref{fig:wkfcad}). Image regularisation focuses on formatting the data while \ac{cade} and \ac{cadx} allow to detect possible lesions and distinguish malignant from non-malignant tumours, respectively.

This paper is organized as follows: \acs{sec}\,\ref{sec:background} deals with general information about human prostate and background about \ac{cap}. Methods regarding \ac{cap} screening and imaging techniques used are also presented as well as an introduction to the \acs{cad} framework. \acs{sec}\,\ref{sec:imaprocfra}\,-\,\ref{sec:dataclassfra} review techniques used in different steps involved in a \acs{cad} work-flow which will be our main contribution. Image regularization framework including pre-processing (\acs{sec}\,\ref{subsec:preprocessing}), segmentation (\acs{sec}\,\ref{subsec:segmentation}) and registration (\acs{sec}\,\ref{subsec:registration}) will be covered as well as \ac{cade} and \ac{cadx} strategies (\acs{sec}\,\ref{sec:dataclassfra}) identified. Results and discussion are reported in \acs{sec}\,\ref{sec:discussion} followed by a concluding section. Deriving from this discussion, we make available to the research community a public online dataset aiming at overcoming some of the drawbacks found when evaluating research in this field.

%%% Local Variables: 
%%% mode: latex
%%% TeX-master: "../g_lemaitre_state_of_the_art"
%%% End: 