\section{Introduction} \label{sec:introduction}

During the last century, physicists focused on constantly innovating in terms of imaging techniques assisting radiologists to overcome different tasks as cancer detection and diagnosis. However, human diagnosis still suffers from low repeatability, synonymous with erroneous detection or interpretations of abnormalities throughout clinical decisions (\cite{Giger2008,Hambrock2013}). These errors are driven by two majors causes (\cite{Giger2008}). On the one hand, observer limitations (e.g., constrained human visual perception, fatigue or distraction) are the principal human issues. On the other hand, the second reason is linked to the clinical cases themselves, for instance due to unbalanced data (number of healthy cases more abundant than malignant cases) or overlapping structures resulting from limitations of imaging techniques.

Computer vision has given rise to many promising solutions, but, instead of focusing on fully automatic computerized systems, researchers have aimed at providing computer image analysis techniques to aid radiologists in their clinical decisions (\cite{Giger2008}). In fact, these investigations brought about both concepts of \ac{cade} and \ac{cadx} grouped under the acronym \acs{cad}. Since those first steps, evidence has shown that \acs{cad} systems enhance the diagnosis performance of radiologists. \cite{Chan1999} reported a significant 4~\% improvement in breast cancer detection, in accordance with later studies (\cite{Dean2006}). Similar conclusions were drawn in the case of lung nodule detection (\cite{Li2004}), colon cancer (\cite{Petrick2008}) and \ac{cap} as well (\cite{Hambrock2013}). \cite{Chan1999} also hypothesized that \acs{cad} systems will be even more efficient assisting inexperienced radiologists to senior radiologists. That hypothesis was tested by \cite{Hambrock2013} and was confirmed in case of \ac{cap} detection. In this particular study, inexperienced radiologists obtained equivalent performance to senior radiologists, both with the help of a \acs{cad} system whereas the accuracy of their diagnosis was significantly poorer without this assistance.

In the late eighties, the first \acs{cad} systems were developed to detect anomalies on chest radiographies and mammograms (\cite{Doi1987,Chan1987,Giger1988}). In the past twenty years, extensive investigations were conducted in the advancement of \acs{cad} systems, migrating from intensive time consuming algorithms performed on reduced number of cases to ``fast'' processing on a large medical dataset. These works were focused on diverse organ cancer diagnosis making use of numerous imaging modalities: micro-calcification detection in breast mammography (\cite{Rangayyan2007,Elter2009}) and \ac{us} imaging (\cite{Cheng2010}), lung nodules detection based on \ac{ct} (\cite{Chan2008,Suzuki2012}), colon tumours detection (\cite{Suzuki2012}) and melanoma detection using dermoscopy imaging (\cite{Korotkov2012}). Noting the abundance of diverse \acs{cad} systems, these fields achieved a certain maturity which can be explained by the imaging techniques employed. Indeed, x-rays, \ac{us} as well as \ac{ct} are medical imaging techniques developed all before the 1970s and were subject to intensive research.

In contradiction with the aforementioned statement, \ac{cap} detection using \acs{cad} is a young technology due to the fact that \ac{mri} is the keystone medical imaging technique (\cite{Hegde2013}). Four distinct \ac{mri} modalities are employed in \acs{cad} for \ac{cap} and were mainly developed after the mid-1990s: (i) \ac{t2w} \ac{mri} (\cite{Hricak1983}), (ii) \ac{dce} \ac{mri} (\cite{HuchBoni1995}), (iii) \ac{mrsi} (\cite{Kurhanewicz1996}) and (iv) \ac{dw} \ac{mri} (\cite{Scheidler1999}). It can be noted that these techniques came into existence relatively recently mainly due to technological progress. In addition, the increase of magnetic field strength and the development of endorectal coil, both improved image spatial resolution (\cite{Swanson2001}) needed to perform more accurate diagnosis. It is for this matter that development of \acs{cad} for \ac{cap} is lagging behind the other fields stated above.

The first study using \ac{mri} as inputs of \acs{cad} system was published ten years ago by \cite{Chan2003}. Despite this, no less than fifty studies have been reviewed for this survey since that seminal work. To the best of our knowledge, there is no review in the literature regarding the advancement of \acs{cad} systems devoted specifically to \ac{cap} detection and diagnosis. Thus, our aim with this survey is threefold: (i) provide an overview of developed \acs{cad} systems for \ac{cap} detection and diagnosis based on \ac{mri} modalities (ii) assess the different work and (iii) pointing out avenues for future work.

As discussed further in Sect. \ref{subsubsec:CAD}, \acs{cad} systems share a common framework. Stages involved in \acs{cad} work-flow can be categorized into six distinctive processes: (i) pre-processing, (ii) segmentation, (iii) registration, (iv) feature detection, (v) feature selection and extraction and (vi) classification. The first three stages are used to enhance data as well as to extract regions of interest and, in the case of multi-modal sources, to merge information of those heterogeneous sources in a joint reference system. The last three categories deal with pattern recognition, machine learning and data mining notions and more precisely with the data classification problem. First, information is detected from the different data sources and a subset of relevant features is selected and/or extracted. Then, this meaningful data will then be classified in order to provide the probability of malignancy of the area of interest and will assist radiologists in their diagnosis decisions (see Fig. \ref{fig:wkfcad}).

This paper is organized as follows: Sect. \ref{sec:background} deals with general information about human prostate and background about \ac{cap}. Methods regarding \ac{cap} screening and imaging techniques used are also presented as well as an introduction on the \acs{cad} framework. Sections \ref{sec:imaprocfra} - \ref{sec:dataclassfra} review techniques used in different steps involved in a \acs{cad} work-flow which will be our main contribution. Image regularization framework including pre-processing (Sect. \ref{subsec:preprocessing}), segmentation (Sect. \ref{subsec:segmentation}) and registration (Sect. \ref{subsec:registration}) will be covered as well as the image classification framework whose feature detection (Sect.\ref{subsec:featuredetection}), feature selection and extraction (see Sect. \ref{subsec:featureselectionextraction}) and feature classification (Sect. \ref{subsec:classification}). Results and discussion are reported in Sect. \ref{sec:discussion} followed by a concluding section.