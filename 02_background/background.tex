\section{Background} \label{sec:background}
This section provides an overview of \ac{cap} as well as its detection and diagnosis. \ac{mri} plays an important role in improving the current strategy and a more detailed description of \ac{mri} modalities is given. Furthermore, a discussion regarding the aim of \ac{cad} systems is also given.

\subsection{Prostate carcinoma}\label{subsec:procar}

\ac{cap} has been reported on a worldwide scale to be the second most frequently diagnosed cancer of men accounting for $13.6 \%$~\cite{Ferlay2010}. Statistically, in 2008, the number of new diagnosed cases was estimated to be $899,000$ with no less than $258,100$ deaths~\cite{Ferlay2010}. In United States, aside from skin cancer, \ac{cap} was declared to be the most commonly diagnosed cancer among men, implying that approximately one in six men will be diagnosed with \ac{cap} during their lifetime and one in thirty-six will die from this disease causing \ac{cap} to be the second most common cause of cancer death among men~\cite{Siegel2013,Society2013}.

Despite active research to determine the causes of prostate cancer, a fuzzy list of risk factors has been established~\cite{Society2010}. The etiology was linked to the following factors~\cite{Society2010}: (i) family history~\cite{Giovannucci2007,Steinberg1990}, (ii) genetic factors~\cite{Freedman2006,Amundadottir2006,Agalliu2009}, (iii) race-ethnicity~\cite{Giovannucci2007,Hoffman2001}, (iv) diet~\cite{Giovannucci2007,Ma2009,Alexander2010}, and (v) obesity~\cite{Giovannucci2007,Rodriguez2007}. This list of risk factors alone cannot be used to diagnose CaP and in this way, screening enables early detection and treatment.

\ac{cap} growth is characterized by two main types of evolution~\cite{Strum2005}: slow and fast. The slow-growing tumours, accounting for up to 85 \% of all \acp{cap}~\cite{Lu-Yao2009}, progress slowly and usually stay confined to the prostate gland. For such cases, treatment can be substituted with active surveillance. In contrast, the second variant of \acp{cap} develops rapidly and metastasises from prostate gland to other organs, primarily the bones~\cite{Oster2013}. Bone metastases, being an incurable disease, significantly affect the morbidity and mortality rate~\cite{Ye2007}. Hence, the  results of the surveillance have to be trustworthy in order to distinguish aggressive from slow-growing \ac{cap}.

\ac{cap} is more likely to develop in specific regions of the prostate. In that respect, around 70-80 \% of \acp{cap} originate in \ac{pz} whereas 10-20 \% in \ac{tz}~\cite{Carrol1987,McNeal1988,Stamey1998}. Only about 5 \% of \acp{cap} occur in \ac{cz}~\cite{McNeal1988,Cohen2008}. However, those cancers appear to be more aggressive and more likely to invade other organs due to their location~\cite{Cohen2008}.

\subsection{\ac{cap} screening and imaging techniques}

\subsubsection{Current \ac{cap} screening}\label{subsubsec:curscr}

Current \ac{cap} screening consists of three different stages. First, \ac{psa} control is performed to distinguish between low and high risk \ac{cap}. Then, for confirmation, samples are taken during \ac{trus} biopsy of the prostate and finally analysed to evaluate the prognosis and the stage of \ac{cap}. Although \ac{psa} screening has been shown to improve early detection of \ac{cap}~\cite{Chou2011}, its lack of reliability motivates further investigations using \ac{mri}~\cite{Andriole2009,Schroeder2012, Hugosson2010}.

Hence, new screening methods should be developed with improved specificity of detection as well as more accurate risk assessment (aggressiveness and progression). Current research is focused on identifying new biological markers to replace \ac{psa}-based screening~\cite{Bourdoumis2010,Morgan2011,Brenner2013}. Until such research comes to fruition, these needs can be met through active-surveillance strategy using multi-parametric \ac{mri} techniques~\cite{Hoeks2011,Moore2013}. A \acs{cad} system based on \ac{mri}, which is an area of active research and forms the focus of this paper, can be incorporated into this screening strategy allowing a more systematic and rigorous follow-up.

Another weakness of the current screening strategy lies in the fact that \ac{trus} biopsy does not provide trustworthy results. Due to its ``blind'' nature imposed by the a random sampling strategy, there is a chance of missing aggressive tumours or detecting microfocal ``cancers'', which influences the aggressiveness-assessment~\cite{Noguchi2001}. As a consequence, over-diagnosis is estimated at up to 30 \%~\cite{Haas2007}, while missing clinically significant \ac{cap} is estimated at up to 35 \%~\cite{Taira2010}. In an effort to solve both issues, alternative biopsy approaches have been explored. \ac{mri}/\ac{us}-guided biopsy has been shown to outperform standard \ac{trus} biopsy~\cite{Delongchamps2013}. There, multimodal \ac{mri} images are fused with \ac{us} images in order to improve localization and aggressiveness assessment to carry out biopsies. Human interaction plays a major role in biopsy sampling which can lead to low repeatability; by reducing potential human errors at this stage, the \acs{cad} framework can be used to improve repeatability of examination.

\ac{cap} detection and diagnosis benefit from the use of \acs{cad} and \ac{mri} techniques. In the following sections, these techniques will be presented in addition to an overview of \acs{cad} for \ac{cap}.

\subsubsection{\ac{mri} imaging techniques}\label{subsubsec:mrimrsi}
Unlike \ac{trus} biopsy, 3.0 Tesla \ac{mri} examination is a non-invasive protocol and has been shown to be the most accurate and harmless technique currently available~\cite{Turkbey2012}. In this section, we review different \ac{mri} modalities developed for \ac{cap} detection and diagnosis. Features used by the radiologists in their daily diagnosis task will receive particular attention together with their drawbacks. Moreover, these features commonly form the basis for developing analytic tools and automatic algorithms. However, we refer the reader to \acs{sec}\,\ref{subsec:featuredetection} for more details on automatic feature detection methods since they are part and parcel of the \acs{cad} framework. An exhaustive review regarding the different modalities as well as the characteristic of each of them is presented in~\cite{Barentsz2012}.

\ac{t2w} \ac{mri} was the first \ac{mri} sequence used to perform \ac{cap} diagnosis using \ac{mri}~\cite{Hricak1983}. Nowadays is a common practice for \ac{cap} detection, localization and staging. This imaging technique is well suited to render zonal anatomy of the prostate~\cite{Barentsz2012}. The features representative of \ac{cap} are indicated in \acs{tab}~\ref{tab:modmri}. In spite of the usefulness of these features, the \ac{t2w} modality lacks reliability in some aspects~\cite{Kirkham2006,Hoeks2011}. Sensitivity is affected by the difficulties in detecting cancers in \ac{cg}~\cite{Kirkham2006} while specificity rate is highly affected by outliers~\cite{Hricak1987,Quint1991,Scheidler1999,Cruz2002,Barentsz2012}.

However, T$_2$ values alone have been shown to be more discriminative~\cite{Liu2011} and highly correlated with citrate concentration, a biological marker in \ac{cap}~\cite{Liney1996,Liney1997}. Similar to \ac{t2w} \ac{mri}, T$_2$ values associated with \ac{cap} are significantly lower than those of healthy tissues~\cite{Liney1996,Gibbs2001}.

\ac{dce} \ac{mri} is an imaging technique which exploits the vascularity characteristic of tissues~\cite{Verma2012}. % An example of the signals obtained during the \ac{dce} \ac{mri} analysis are depicted in Fig.~\ref{fig:dceana}. 
Three different approaches exist to analyse these signals with the aim of tagging them as corresponding to either normal or malignant tissues: qualitative analysis is based on assessment of the signal shape~\cite{Hoeks2011}; quantitative approaches consist of inferring pharmocokinetic parameter values~\cite{Tofts2010}; and semi-quantitative methods which rely on shape characterization using mathematical modelling to extract a set of parameters~\cite{Hoeks2011,Verma2012}. It was shown that semi-quantitative and quantitative methods improve localization of \ac{cap} when compared with qualitative methods~\cite{Rosenkrantz2013}. \Acl{tab}~\ref{tab:modmri} gives an overview of the features used during \ac{dce} \ac{mri} analysis. \Acl{sec}~\ref{subsubsec:fddce} provides a full description of quantitative and semi-quantitative approaches. \ac{dce} \ac{mri} combined with \ac{t2w} \ac{mri} has shown to enhance sensitivity compared to \ac{t2w} \ac{mri} alone~\cite{Jager1997,Kim2005,Schlemmer2004,Zelhof2009}. Despite this fact, \ac{dce} \ac{mri} possesses some drawbacks. Due to its ``dynamic'' nature, patient motions during the image acquisition may lead to spatial misregistration of the image set~\cite{Verma2012}. Furthermore, it has been suggested that malignant tumours are difficult to distinguish from prostatitis located in \ac{pz} and \ac{bph} located in \ac{cg}~\cite{Hoeks2011,Verma2012} as these two pairs of tissues tend to have similar appearances. Later studies have shown that \acp{cap} in \ac{cg} do not always manifest in homogeneous fashion. Indeed, tumours in this zone can present both hypo-vascularization and hyper-vascularization which illustrates the challenge of \ac{cap} detection in \ac{cg}~\cite{Niekerk2013}.

\ac{dw} \ac{mri} is the most recent MRI imaging technique aiming at \ac{cap} detection and diagnosis~\cite{Scheidler1999}. This modality exploits the variations in the motion of water molecules in different tissues~\cite{LeBihan1988,Koh2007}. \Acl{tab}~\ref{tab:modmri} summarizes the markers used in \ac{dw} \ac{mri} to distinguish \ac{cap}. However, some tissues in \ac{cg} can look similar to \ac{cap} with higher \ac{si}~\cite{Barentsz2012}. Diagnosis using \ac{dw} \ac{mri} combined with \ac{t2w} \ac{mri} has shown a significant improvement compared with \ac{t2w} \ac{mri} alone and provides highly contrasted images~\cite{Shimofusa2005,Padhani2011,Choi2007}. As drawbacks, this modality suffers from poor spatial resolution and low specificity~\cite{Choi2007}.

With a view to eliminate these drawbacks, radiologists are extracting quantitative maps from \ac{dw} \ac{mri} which is known as the \ac{adc} map. The \ac{adc} coefficient is considered as a ``pure'' diffusion coefficient. This coefficient varies inversely to \ac{dw} \ac{mri} images~\cite{Barentsz2012}. % as depicted in Fig.~\ref{subfig:adc}. 
Similar to the gain achieved by \ac{dw} \ac{mri}, diagnosis using \ac{adc} map combined with \ac{t2w} \ac{mri} significantly outperforms \ac{t2w} \ac{mri} alone~\cite{Doo2012,Choi2007}. Moreover, it has been shown that \ac{adc} is correlated with \ac{gs}~\cite{Hambrock2011, Itou2011, Peng2013}. However, some tissues of the \ac{cg} zone mimic \ac{cap} with low-\ac{si}~\cite{Kirkham2006} and image distortion can arise due to haemorrhage~\cite{Choi2007}. It has also been noted that a high variability of the \ac{adc} occurs between different patients making it difficult to define a static threshold to distinguish \ac{cap} from non-malignant tumours~\cite{Choi2007}.

\ac{cap} induces metabolic changes in the prostate compared with healthy tissue. Thus, \ac{cap} detection can be carried out by tracking changes of metabolite concentration in prostate tissue. \ac{mrsi} is an \ac{nmr}-based technique which generates spectra of relative metabolite concentration in \iac{roi}. In order to track changes of metabolite concentration, it is important to know which metabolites are associated with \ac{cap}. To address this question, clinical studies identified three biological markers: (i) citrate, (ii) choline and (iii) polyamines composed mainly of spermine, and in less abundance of spermidine and putrescine~\cite{Awwad2012,Costello2006,Giskeodegard2013}. The variations of the metabolites are reported in \ac{tab}~\ref{tab:modmri}. \ac{mrsi} allows examination with high specificity and sensitivity compared to other \ac{mri} modalities~\cite{Choi2007}. Furthermore, it has been shown that combining \ac{mrsi} with \ac{mri} improves detection and diagnosis performance~\cite{Scheidler1999a,Kaji1998,Vilanova2009}. Citrate and spermine concentrations are inversely correlated with the \ac{gs} allowing to distinguish low from high grade \acp{cap}~\cite{Giskeodegard2013}. However, choline concentration does not provide the same properties~\cite{Giskeodegard2013}. Unfortunately, \ac{mrsi} also presents several drawbacks. First, \ac{mrsi} acquisition is time consuming which prevents this modality from being used in daily clinical practise~\cite{Barentsz2012}. In addition, \ac{mrsi} suffers from low spatial resolution due to the fact that \ac{snr} is linked to the voxel size. However, this issue is addressed by developing new scanners with higher magnetic field strengths such as 7.5 T~\cite{Giskeodegard2013}. Finally, a high variability of the relative concentrations between patients has been observed~\cite{Choi2007}. The same observation was made depending on the zones studied (cf., \ac{pz}, \ac{cg}, base, mid-gland, apex)~\cite{Walker2010,Lemaitre2011}. Due to this variability, it is difficult to use a fixed threshold in order to differentiate \ac{cap} from healthy tissue.

\subsubsection{Computer-aided systems for \ac{cap}: \ac{cade} - \ac{cadx}} \label{subsubsec:CAD}

As previously mentioned in the introduction (see \acs{sec}\,\ref{sec:introduction}), \acp{cad} are developed to advise and backup radiologists in their tasks of \ac{cap} detection and diagnosis, but not to provide fully automatic decisions~\cite{Giger2008}. \acp{cad} can be divided into two different sub-groups either as \ac{cade}, with the purpose to highlight probable lesions in \ac{mri} images, or \ac{cadx}, which focuses on differentiating malignant from non-malignant tumours~\cite{Giger2008}. Moreover, an intuitive approach, motivated by developing a framework combining detection-diagnosis, is to mix both \ac{cade} and \ac{cadx} by using the output of the former mentioned as a input of the latter named. Although the outcomes of these two systems should differ, the framework of both \ac{cad} systems is similar. A general \ac{cad} work-flow is presented in \acs{fig}\,\ref{fig:wkfcad}.

\ac{mri} modalities mentioned in \acs{sec}\,\ref{subsubsec:mrimrsi} are used as inputs of \ac{cad} for \ac{cap}. The images acquired from the different modalities show a large variability between patients: the prostate organ can be located at different positions in images (e.g., patient motion, variation of acquisition plan), and the \ac{si} can be corrupted with noise or artefacts during the acquisition process (eg., magnetic field inhomogeneity, use of endorectal coil). To address these issues, the first stage of \ac{cad} is to pre-process multiparametric \ac{mri} images to reduce noise, remove artefacts and standardize the \ac{si}. As most of the later processes will be only focused on the prostate. It is necessary to segment the prostate in each \ac{mri}-modality to define it as \iac{roi}. However, data may suffer from misalignment due to patient motion or different acquisition parameters. Therefore, a registration step is usually performed so that all the previously segmented \ac{mri} images will be in the same reference frame. Registration and segmentation steps can be swapped depending on the strategy chosen.

Some studies do not fully apply the methodology depicted in \acs{fig}\,\ref{fig:wkfcad}. Details about those can be found in \acs{tab}~\ref{tab:sumpap}. Some studies proposed methods in which inputs are the \ac{mri} raw data in order to demonstrate the robustness of their approaches to noise or artefacts. In some cases, prostate segmentation is performed manually as well as registration. It is also sometimes assumed that no patient motions occur during the acquisition procedure, removing the need of registering the multiparametric \ac{mri} images.

Once the data are regularized, it becomes possible to extract features and classify these data to obtain either the location of possible lesions (\ac{cade}) or/and the malignancy nature of these lesions (\ac{cadx}).

In \iac{cade} framework, \textit{possible lesions will be segmented automatically} and further used as inputs of a \ac{cadx}. Nevertheless, some works also used a fused \ac{cade}-\ac{cadx} framework in which a voxel-based features are directly used, allowing to obtain the location of the malignant lesions as results. On the other hand, manual lesions segmentation is not considered to be part of \iac{cade}.

\Ac{cadx} is composed of the processes allowing to \textit{distinguish malignant from non-malignant tumours}. In the studies reviewed, \ac{cap} malignancy is defined using the grade of the \ac{gs} determined after post-biopsy or prostatectomy. As presented in Fig.\,\ref{fig:wkfcad}, \ac{cadx} is usually composed of the three common steps used in classification framework: (i) features detection, (ii) features extraction/selection and (iii) features classification.

\subsection{Literature classification}

The \ac{cad} review is organized using the methodology presented in \acs{fig}\,\ref{fig:wkfcad}. Methods embedded in the image regularization framework are presented initially to subsequently focus on the image classification framework, being divided into \ac{cade} and \ac{cadx}. \Acl{tab}~\ref{tab:sumpap} summarizes the forty-two different \ac{cad} studies reviewed in this paper. The first set of information reported is linked to the data acquisition such as the number of patients included in the study, the modalities acquired as well as the strength of the field of the scanner used. Subsequently, information about the prostate zones considered in the \ac{cad} analysis (\ac{pz} or \ac{cg}) are reported since that detecting \ac{cap} in the \ac{cg} is a more challenging problem and has received particular attention only in recent publications.

%%% Local Variables: 
%%% mode: latex
%%% TeX-master: "../g_lemaitre_state_of_the_art"
%%% End: 
