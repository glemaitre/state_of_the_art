\section{Background} \label{sec:background}

This section provides an overview of \ac{cap} as well as its detection and diagnosis. Subsequently a discussion of the current screening strategy for \ac{cap} and its drawbacks is presented. \ac{mri} plays an important role in improving the current strategy and a more detailed description of \ac{mri} modalities is given. Furthermore, a discussion regarding the aim of \ac{cad} systems will be given.

\subsection{Prostate carcinoma}\label{subsec:procar}

\ac{cap} has been reported on a worldwide scale to be the second most frequently diagnosed cancer of men accounting for $13.6 \%$ (\cite{Ferlay2010}). Statistically, in 2008, the number of new diagnosed cases was estimated to be $899,000$ with no less than $258,100$ deaths (\cite{Ferlay2010}). In United States, aside from skin cancer, \ac{cap} was declared to be the most commonly diagnosed cancer among men, implying that approximately one in six men will be diagnosed with \ac{cap} during their lifetime and one in thirty-six will die from this disease causing \ac{cap} to be the second most common cause of cancer death among men (\cite{Siegel2013}, \cite{Society2013}).

Despite active research to determine the causes of prostate cancer, a fuzzy list of risk factors has been established (\cite{Society2010}). The etiology was linked to the following factors (\cite{Society2010}): (i) family history (\cite{Giovannucci2007,Steinberg1990}), (ii) genetic factors (\cite{Freedman2006,Amundadottir2006,Agalliu2009}), (iii) race-ethnicity (\cite{Giovannucci2007,Hoffman2001}), (iv) diet (\cite{Giovannucci2007,Ma2009,Alexander2010}), (v) obesity (\cite{Giovannucci2007,Rodriguez2007}). This list of risk factors alone cannot be used to diagnose CaP and in this way, screening enables early detection and treatment.

\ac{cap} growth is characterized by two main types of evolution (\cite{Strum2005}): slow and fast. The slow-growing tumours, accounting for up to 85 \% of all \acp{cap} (\cite{Lu-Yao2009}), progress slowly and usually stay confined to the prostate gland. For such cases, treatment can be substituted with active surveillance. In contrast, the second variant of \acp{cap} develops rapidly and metastasises from prostate gland to other organs, primarily the bones (\cite{Oster2013}). Bone metastases, being an incurable disease, significantly affect the morbidity and mortality rate (\cite{Ye2007}). Hence, the  results of the surveillance have to be trustworthy in order to distinguish aggressive from slow-growing \ac{cap}.

\ac{cap} is more likely to develop in specific regions of the prostate. In that respect, around 70-80 \% of \acp{cap} originate in \ac{pz} whereas 10-20 \% in \ac{tz} (\cite{Carrol1987,McNeal1988,Stamey1998}). Only about 5 \% of \acp{cap} occur in \ac{cz} (\cite{McNeal1988,Cohen2008}). However, those cancers appear to be more aggressive and more likely to invade other organs due to their location (\cite{Cohen2008}).

\subsection{\ac{cap} screening and imaging techniques}

\subsubsection{Current \ac{cap} screening}\label{subsubsec:curscr}

Current \ac{cap} screening consists of three different stages. First, \ac{psa} control is performed to distinguish between low and high risk \ac{cap}. Then, for confirmation, samples are taken during prostate biopsy and finally analysed to evaluate the prognosis and the stage of \ac{cap}. In this section, we present a detailed description of the current screening as well as its drawbacks.

Since its introduction in mid-1980s, \ac{psa} is widely used for \ac{cap} screening (\cite{Etzioni2002}). A higher-than-normal level of \ac{psa} can indicate an abnormality of the prostate either as a \ac{bph} or a cancer (\cite{Hoeks2011}). However, other factors can lead to an increased \ac{psa} level such as prostate infections, irritations, a recent ejaculation or a recent rectal examination (\cite{Parfait2010}).

\Iac{trus} biopsy is carried out for cases which are considered as pathological. At least six different samples are taken randomly from the right and left parts of three three different zones: apex, median and base. These samples are further evaluated using the Gleason grading system (\cite{Gleason1977}). Also, it should be noted that biopsy is an invasive procedure which can result in serious infection or urine retention (\cite{Hara2005,Chou2011}).

Although \ac{psa} screening has been shown to improve early detection of \ac{cap} (\cite{Chou2011}), its lack of reliability motivates further investigations using \ac{mri}. Two reliable studies, carried out in the United States (\cite{Andriole2009}) and in Europe (\cite{Schroeder2012, Hugosson2010}), have attempted to assess the impact of early detection of \ac{cap}, with diverging outcomes (\cite{Chou2011,Heidenreich2013}). The study carried out in Europe\footnote{The \ac{ersspc}  started in the 1990s in order to evaluate the effect of \ac{psa} screening on mortality rate.} concluded that \ac{psa} screening reduces CaP-related mortality by 21-44\% (\cite{Schroeder2012, Hugosson2010}), while the American\footnote{The \ac{plco} cancer screening trial is carried out in the United States and intends to ascertain the effects of screening on mortality rate.} trial found no such effect (\cite{Andriole2009}). However, both studies agree that \ac{psa} screening suffers from low specificity, with an estimated rate of 36 \% (\cite{Schroder2008}). Both studies also agree that over-treatment is an issue: decision making regarding treatment is further complicated by difficulties in evaluating the aggressiveness and progression of \ac{cap} (\cite{Delpierre2013}). 

Hence, new screening methods should be developed with improved specificity of detection as well as more accurate risk assessment (aggressiveness and progression). Current research is focused on identifying new biological markers to replace \ac{psa}-based screening (\cite{Bourdoumis2010,Morgan2011,Brenner2013}). Until such research comes to fruition, these needs can be met through active-surveillance strategy using multi-parametric \ac{mri} techniques (\cite{Hoeks2011,Moore2013}). An \ac{mri}-\acs{cad} system, which is an area of active research and forms the focus of this paper, can be incorporated into this screening strategy allowing a more systematic and rigorous follow-up.

Another weakness of the current screening strategy lies in the fact that \ac{trus} biopsy does not provide trustworthy results. Due to its ``blind'' nature imposed by the a random sampling strategy, there is a chance of missing aggressive tumours or detecting microfocal ``cancers'', which influences the aggressiveness-assessment procedure (\cite{Noguchi2001}). As a consequence, over-diagnosis is estimated at up to 30 \% (\cite{Haas2007}), while missing clinically significant \ac{cap} is estimated at up to 35 \% (\cite{Taira2010}). In an effort to solve both issues, alternative biopsy approaches have been explored. \ac{mri}/\ac{us}-guided biopsy has been shown to outperform standard \ac{trus} biopsy (\cite{Delongchamps2013}). There, multimodal \ac{mri} images are fused with \ac{us} images in order to improve localization and aggressiveness assessment to carry out biopsies. Human interaction plays a major role in biopsy sampling which can lead to low repeatability; by reducing potential human errors at this stage, the \acs{cad} framework can be used to improve repeatability of examination.

\ac{cap} detection and diagnosis benefit from the use of \acs{cad} and \ac{mri} techniques. In the following sections, these techniques will be presented in addition to an overview of \acs{cad} for \ac{cap}.

\subsubsection{\ac{mri} imaging techniques}\label{subsubsec:mrimrsi}

\ac{mri} provides promising imaging techniques to overcome the previous mentioned drawbacks. Unlike \ac{trus} biopsy, \ac{mri} examination is a non-invasive protocol and has been shown to be the most accurate and harmless technique currently available (\cite{Turkbey2012}). In this section, we review different \ac{mri} modalities deveoped for \ac{cap} detection and diagnosis. Features used by the radiologist in their daily diagnosis task will receive particular attention together with their drawbacks. Moreover, these features commonly form the basis for developing analytic tools and automatic algorithms. However, we refer the reader to \acs{sec} \ref{subsec:featuredetection} for more details on automatic feature detection methods since they are part and parcel of the \acs{cad} framework. Table \ref{tab:modmri} provides an overview of the following discussion.

% We are using enumerate with a small margin and some indent to organize our thoughts by paragraphs.
\setenumerate{listparindent=\parindent,itemsep=10px}
\setlist{noitemsep}
\begin{enumerate}[leftmargin=*]

  % T2W MRI
\item[$-$] \textbf{\textit{\ac{t2w} \ac{mri}:}} \ac{t2w} \ac{mri} was the first \ac{mri}-modality used to perform \ac{cap} diagnosis using \ac{mri} (\cite{Hricak1983}). Nowadays, radiologists make use of it for \ac{cap} detection, localization and staging purposes. This imaging technique is well suited to render zonal anatomy of the prostate (\cite{Barentsz2012}). 

  \ac{pz} and \ac{cg} tissues are well perceptible in these images. The former is characterized by an intermediate/high-\ac{si} while the latter is depicted by a low-\ac{si} (\cite{Hricak1987}). An example of a healthy prostate is shown in Fig.~\ref{subfig:t2whealthy}.

  In \ac{pz}, round or ill-defined low-SI masses are synonymous with \acp{cap} (\cite{Hricak1983}) as shown in Fig.~\ref{subfig:t2wcancerpz}. Detecting \ac{cap} in \ac{cg} is more challenging. In fact both normal \ac{cg} tissue and malignant tissue, have a low-\ac{si} in \ac{t2w} \ac{mri} reinforcing difficulties to distinguish between them. However, \acp{cap} in \ac{cg} appear often as homogeneous mass possessing ill-defined edges with lenticular or ``water-drop'' shapes (\cite{Akin2006, Barentsz2012}) as depicted in Fig.~\ref{subfig:t2wcancercg}. 

  \ac{cap} aggressiveness was shown to be inversely correlated with \ac{si}. Indeed, \acp{cap} assessed with a \ac{gs} of 4-5 implied lower \ac{si} than the one with a \ac{gs} of 2-3 (\cite{Wang2008}).

  In spite of the availability of these useful and encouraging features, the \ac{t2w} modality lacks reliability (\cite{Kirkham2006,Hoeks2011}). Sensitivity is affected by the difficulties in detecting cancers in \ac{cg} (\cite{Kirkham2006}) while specificity rate is highly affected by outliers (\cite{Barentsz2012}). In fact, various conditions emulate patterns of \ac{cap} such as \ac{bph}, post-biopsy haemorrhage, atrophy, scars and post-treatment (\cite{Hricak1987,Quint1991,Scheidler1999,Cruz2002,Barentsz2012}). These issues can be partly addressed using more innovative and advanced modalities such as later presented.

  % T2 Map
\item[$-$] \textbf{\textit{T$_2$ Map:}} As previously mentioned, \ac{t2w} \ac{mri} modality shows low sensitivity due to various effects (\cite{Hegde2013}). However, T$_2$ values alone have been shown to be more discriminative (\cite{Liu2011}) and highly correlated with citrate concentration, a biological marker in \ac{cap} (\cite{Liney1996,Liney1997}). 

  The \Ac{fse} sequence has been shown to be particularly well suited in order to build a T$_2$ map and obtain accurate T$_2$ values (\cite{Liney1996a}).

  Similar to \ac{t2w} \ac{mri}, T$_2$ values associated with \ac{cap} are significantly lower than those of healthy tissues (\cite{Liney1996,Gibbs2001}).

  % DCE MRI
\item[$-$] \textbf{\textit{\ac{dce} \ac{mri}:}} \ac{dce} \ac{mri} is an imaging technique which exploits the vascularity characteristic of tissues. Contrast media, usually gadolinium-based, is injected intravenously into the patient. The media extravasates from vessels to the \ac{ees} and is then released back into the vasculature before being eliminated by the kidneys (\cite{Gribbestad2005}). Furthermore, the diffusion speed of the contrast agent may vary due to several parameters: (i) the permeability of the micro-vessels, (ii) their surface area and (iii) the blood flow (\cite{Padhani2002}).

  \ac{dce} \ac{mri} is based on an acquisition of a set of \ac{t1w} \ac{mri} images over time. the Gadolinium-based contrast agent shortens T$_1$ relaxation time enhancing contrast in \ac{t1w} \ac{mri} images. The aim is to post-analyse the pharmacokinetic behaviour of the contrast media concentration in prostate tissues (\cite{Verma2012}). The image analysis is carried out in two dimensions: (i) in the spatial domain on a pixel-by-pixel basis and (ii) in the time domain corresponding to the consecutive images acquired with the \ac{mri}. Thus, for each spatial location, a signal linked to contrast media concentration is measured as shown in Fig.~\ref{fig:dceana} (\cite{Tofts2010}). \acp{cap} are characterized by a signal having an earlier and faster enhancement as well as an earlier wash-out (cf., the rate of the contrast agent flowing out of the tissue) (see Fig.~\ref{subfig:dce}) (\cite{Verma2012}). Three different approaches exist to analyse these signals with the aim of tagging them as corresponding to either normal or malignant tissues. Qualitative analysis is based on assessment of the signal shape (\cite{Hoeks2011}). Quantitative approaches consist of inferring pharmocokinetic parameter values (\cite{Tofts2010}). Those parameters are part of mathematical-pharmacokinetic models which are directly based on physiological exchanges between vessels and \ac{ees}. The last family of methods mix both approaches and are grouped together under the heading of semi-quantitative methods. They rely on shape characterization using mathematical modelling to extract a set of parameters. These parameters will be discussed in a later section (see Fig.~\ref{fig:dceparam}) (\cite{Hoeks2011,Verma2012}). It was shown that semi-quantitative and quantitative methods improve localization of \ac{cap} when compared with qualitative methods (\cite{Rosenkrantz2013}). Section \ref{subsubsec:fddce} provides a full description of quantitative and semi-quantitative approaches.

  \ac{dce} \ac{mri} combined with \ac{t2w} \ac{mri} has shown to enhance sensitivity compared to \ac{t2w} \ac{mri} alone (\cite{Jager1997,Kim2005,Schlemmer2004,Zelhof2009}). Despite this fact, \ac{dce} \ac{mri} possesses some drawbacks. Due to its ``dynamic'' nature, patient motions during the image acquisition may lead to spatial misregistration of the image set (\cite{Verma2012}). Furthermore, it has been suggested that malignant tumours are difficult to distinguish from prostatitis located in \ac{pz} and \ac{bph} located in \ac{cg} (\cite{Hoeks2011,Verma2012}) as these two pairs of tissues tend to have similar appearances. Later studies have shown that \acp{cap} in \ac{cg} do not always manifest in homogeneous fashion. Indeed, tumours in this zone can present both hypo-vascularization and hyper-vascularization which illustrates the challenge of \ac{cap} detection in \ac{cg} (\cite{Niekerk2013}).

  % DWI MRI
\item[$-$] \textbf{\textit{\ac{dw} \ac{mri}:}} As previously mentioned in the introduction, \ac{dw} \ac{mri} is the most recent MRI imaging technique aiming at \ac{cap} detection and diagnosis (\cite{Scheidler1999}). This modality exploits the variations in the motion of water molecules in different tissues (\cite{LeBihan1988,Koh2007}).

  From the \ac{nmr} principle side, \ac{dw} \ac{mri} sequence produces contrasted images due to variation of water molecules motion. The method is based on the fact that the signal in \ac{dw} \ac{mri} images is inversely correlated to the degree of random motion of water molecules (\cite{Huisman2003}). A higher degree of random motion results in a more significant signal loss whereas a lower degree of random motion is synonymous with lower signal loss (\cite{Huisman2003}). Under these conditions, the MRI signal is measured as:

  \begin{equation}
    M_{x,y}\left(t,b\right) = M_{x,y}(0) \exp \left( - \frac{t}{\text{T}_2} \right) S_{\text{ADC}}(b) \ , 
    \label{eq:t2dif}
  \end{equation}

  \begin{equation}
    S_{\text{ADC}}(b) = \exp \left( -b \times \text{ADC} \right) \ ,
    \label{eq:dif}
  \end{equation}

  \noindent where $S_{\text{ADC}}$ refers to signal drop due to diffusion effect, $\text{ADC}$ is the \acl{adc} and $b$ is the attenuation coefficient depending only on gradient pulses parameters: (i) gradient intensity and (ii) gradient duration (\cite{LeBihan1986}).

  By using this formulation, image acquisition with a parameter $b=0$ s.mm$^{-2}$ corresponds to a \ac{t2w} \ac{mri} acquisition. Then, increasing the attenuation coefficient $b$ (cf., increase gradient intensity and duration) enhances the contrast in \ac{dw} \ac{mri} images.

  To summarize, in \ac{dw} \ac{mri} images, \acp{cap} are characterized by high-\ac{si} compared to normal tissues in \ac{pz} and \ac{cg} as shown in Fig.~\ref{subfig:dwi} (\cite{Barentsz2012}). However, some tissues in \ac{cg} can look similar to \ac{cap} with higher \ac{si} (\cite{Barentsz2012}).

  Diagnosis using \ac{dw} \ac{mri} combined with \ac{t2w} \ac{mri} has shown a significant improvement compared with \ac{t2w} \ac{mri} alone and provides highly contrasted images (\cite{Shimofusa2005,Padhani2011,Choi2007}). As drawbacks, this modality suffers from poor spatial resolution and low specificity due to false positive detection (\cite{Choi2007}).

  With a view to eliminate these drawbacks, radiologists are extracting quantitative maps from \ac{dw} \ac{mri}. This imaging technique is presented next.

  % ADC map
\item[$-$] \textbf{\textit{\ac{adc} Map:}} The \ac{nmr} signal measured for \ac{dw} \ac{mri} images is not only affected by diffusion as shown in \acs{eq} \eqref{eq:t2dif}. However, the signal drop (\acs{eq} \eqref{eq:dif}) is formulated such that the only variable is the acquisition parameter $b$ (\cite{LeBihan1986}). The \ac{adc} is considered as a ``pure'' diffusion coefficient and can be extracted to build a quantitative map.

  From \acs{eq} \ref{eq:t2dif}, it is clear that performing multiple acquisitions only varying $b$ will not have any effect on the term  $M_{x,y}(0) \exp \left( - \frac{t}{\text{T}_2} \right)$. Thus, \acs{eq} \ref{eq:t2dif} can be rewritten as:

  \begin{equation}
    S(b) = S_0 \exp \left( -b \times \text{ADC} \right) \ .
    \label{eq:t2adcrew}
  \end{equation}

  To compute the \ac{adc} map, a minimum of two acquisitions are necessary: (i) for $b_0=0$ s.mm$^{-2}$ where the measured signal is equal to $S_0$, and (ii) $b_1>0$ s.mm$^{-2}$ (typically $1000$ s.mm$^{-2}$). Then, the \ac{adc} map can be computed as:

  \begin{equation}
    \text{ADC} = - \frac{\ln \left( \cfrac{S(b_1)}{S_0} \right) }{b_1} \ .
    \label{eq:adcres1}
  \end{equation}

  More accurate computation of the \ac{adc} map can be obtained by performing several acquisitions with different values for the parameter $b$ and performing a semi-logarithmic linear fitting using the model presented in \acs{eq} \eqref{eq:t2adcrew}.

  Regarding the appearance of the \ac{adc} maps, it was previously stated that by increasing the value of $b$, the signal of \ac{cap} tissue increases significantly. From \acs{eq} \eqref{eq:adcres1}, it can be shown that tissue appearance in the ADC map will be the inverse of \ac{dw} \ac{mri} images. Then, \ac{cap} tissue is associated with low-\ac{si} whereas healthy tissue appears brighter as depicted in Fig.~\ref{subfig:adc} (\cite{Barentsz2012}).

  Similar to the gain achieved by \ac{dw} \ac{mri}, diagnosis using \ac{adc} map combined with \ac{t2w} \ac{mri} significantly outperforms \ac{t2w} \ac{mri} alone (\cite{Doo2012,Choi2007}). Moreover, it has been shown that \ac{adc} is correlated with \ac{gs} (\cite{Hambrock2011, Itou2011, Peng2013}).

  However, some tissues of the \ac{cg} zone mimic \ac{cap} with low-\ac{si} (\cite{Kirkham2006}) and image distortion can arise due to haemorrhage (\cite{Choi2007}). It has also been noted that a high variability of the \ac{adc} occurs between different patients making it difficult to define a static threshold to distinguish \ac{cap} from non-malignant tumours (\cite{Choi2007}).

  % MRSI
\item[$-$] \textbf{\textit{\ac{mrsi}:}} \ac{cap} induces metabolic changes in the prostate compared with healthy tissue. Thus, \ac{cap} detection can be carried out by tracking changes of metabolite concentration in prostate tissue. \ac{mrsi} is an \ac{nmr}-based technique which generates spectra of relative metabolite concentration in \iac{roi}.

  In order to track changes of metabolite concentration, it is important to know which metabolites are associated with \ac{cap}. To address this question, clinical studies identified three biological markers: (i) citrate, (ii) choline and (iii) polyamines composed mainly of spermine, and in less abundance of spermidine and putrescine (\cite{Awwad2012,Costello2006,Giskeodegard2013}). 

  An increased concentration of choline associated with a decreased concentration of citrate and spermine are related to the presence of \ac{cap} (\cite{Awwad2012,Costello2006,Graaf2000,Giskeodegard2013}).

  To determine the concentration of these biological markers, one has to focus on the \ac{mrsi} modality. In each spectrum acquired, each peak is associated with a particular metabolite and the area under each peak corresponds to the relative concentration of this metabolite (see Fig.~\ref{fig:mrsi}) (\cite{Parfait2010}).

  Hence, frequencies of interest in regard to \ac{cap} detection and diagnosis should correspond to the earlier mentioned metabolites. Choline and spermine are represented by a single peak at respectively 3.21 ppm and 3.11 ppm (\cite{Verma2010}). Due to the coupling effect, citrate is represented by three or four peaks depending on the magnetic field strength. Citrate ranges from 2.47 ppm to 2.81 ppm with a central frequency at 2.64 ppm (\cite{Verma2010}). Then, relative concentrations of these metabolites are obtained by computing the area under the curve of the spectrum between the lower and upper frequency limits of each peak (see Fig.~\ref{fig:mrsi}). It can be noted that a creatine peak is located at 3.02 ppm and the three metabolite peaks tend to be merged together at clinical magnetic field strengths (see Fig.~\ref{fig:mrsi}) (\cite{Hoeks2011,Graaf2000}).

  \ac{mrsi} allows examination with high specificity and sensitivity compared to other \ac{mri} modalities (\cite{Choi2007}). Furthermore, it has been shown that combining \ac{mrsi} with \ac{mri} improves detection and diagnosis performance (\cite{Scheidler1999a,Kaji1998,Vilanova2009}). Citrate and spermine concentrations are inversely correlated with the \ac{gs} allowing us to distinguish low from high grade \acp{cap} (\cite{Giskeodegard2013}). However, choline concentration does not provide the same properties (\cite{Giskeodegard2013}).

  Unfortunately, \ac{mrsi} also presents several drawbacks. First, \ac{mrsi} acquisition is time consuming which prevents this modality from being used in daily clinical practise (\cite{Barentsz2012}). In addition, \ac{mrsi} suffers from low spatial resolution due to the fact that \ac{snr} is linked to the voxel size. However, this issue is addressed by developing new scanners with higher magnetic field strengths such as 7.5 T (\cite{Giskeodegard2013}). Finally, a high variability of the relative concentrations between patients has been observed (\cite{Choi2007}). The same observation was made depending on the zones studied (cf., \ac{pz}, \ac{cg}, base, mid-gland, apex) (\cite{Walker2010,Lemaitre2011}). Due to this variability, it is difficult to use a fixed thresholds in order to differentiate \ac{cap} from healthy tissue.

\end{enumerate}

\subsubsection{Computer-aided systems for \ac{cap}: \ac{cade} - \ac{cadx}} \label{subsubsec:CAD}

As previously mentioned in the introduction (see \acs{sec} \ref{sec:introduction}), \acp{cad} are developed to advise and backup radiologists in their tasks of \ac{cap} detection and diagnosis; \acp{cad} are not aimed to provide fully automatic decisions (\cite{Giger2008}). \acp{cad} can be divided into two different sub-groups either as \ac{cade}, with the purpose to highlight probable lesions in \ac{mri} images, or \ac{cadx}, which focuses on differentiating malignant from non-malignant tumours (\cite{Giger2008}). Moreover, an intuitive approach, motivated by developing a framework combining detection-diagnosis, is to mix both \ac{cade} and \ac{cadx} by using the output of the former mentioned as a input of the latter named. Although the outcomes of these two systems should differ, the framework of both \ac{cad} systems is similar. The \ac{cad} work-flow is presented in \acs{fig} \ref{fig:wkfcad}.

\ac{mri} modalities mentioned in \acs{sec} \ref{subsubsec:mrimrsi} are used as inputs of \ac{cad} for \ac{cap}. The images acquired from the different modalities show a large variability between patients: the prostate organ can be located at different positions in images (e.g., patient motion, variation of acquisition plan), and the \ac{si} can be corrupted with noise or artefacts during the acquisition process (eg., magnetic field inhomogeneity, use of endorectal coil). To address these issues, the first stage of \ac{cad} is to pre-process multiparametric \ac{mri} images to reduce noise, remove artefacts and standardize the \ac{si}. As most of the later processes will be only focused on the prostate. It is necessary to segment the prostate in each \ac{mri}-modality to define it as \iac{roi}. However, data may suffer from misalignment due to patient motion or different acquisition parameters. Therefore, a registration step is usually performed so that all the previously segmented \ac{mri} images will be in the same reference frame. Registration and segmentation steps can be swapped depending on the strategy chosen.

Some studies do not fully apply the methodology depicted in \acs{fig} \ref{fig:wkfcad}. Details about those can be found in \acs{tab} \ref{tab:sumpap}. Some studies preferred to work directly with raw data in order to demonstrate the robustness of their approaches to noise or artefacts. In some cases, prostate segmentation is performed manually as well as registration. It is also sometimes assumed that no patient motions occur during the acquisition procedure, removing the need of registering the multiparametric \ac{mri} images.

Once the data are regularized, it becomes possible to extract features and classify these data to obtain either the location of possible lesions (\ac{cade}) or/and the malignancy nature of these lesions (\ac{cadx}).

In \iac{cade} framework, \textbf{\textit{possible lesions will be segmented automatically}} and further used as inputs of a \ac{cadx}. Nevertheless, this is not the only case that \ac{cade} frameworks encountered. Some works also used a fused \ac{cade}-\ac{cadx} framework in which a voxel-based delineation approach is used, allowing to obtain the location of the malignant lesions as results. On the other hand, manual lesions segmentation are not considered to be part of \iac{cade}.

\Ac{cadx} is composed of the processes allowing to \textbf{\textit{distinguish malignant from non-malignant tumours}}. In the studies reviewed, \ac{cap} malignancy is defined using the grade of the \ac{gs} determined after post-biopsy or prostatectomy. As presented in Fig. \ref{fig:wkfcad}, \ac{cadx} is usually composed of the three common steps used in classification framework: (i) features detection, (ii) features extraction/selection and (iii) features classification.

As pointed out in the introduction, performance of \ac{cap} detection and diagnosis are affected by observer interpretation and limitations (\cite{Giger2008,Hambrock2013}). \ac{cad} offers a possible solution in order to reduce this variability. As mentioned in the introduction, the effects of \ac{cad} on the observer performance has been studied (\cite{Hambrock2013}), with results showing that \acp{cad} benefit to less-experienced radiologist to perform similarly as experienced radiologist in their tasks (\cite{Hambrock2013}). 

\subsection{Literature classification}

The \ac{cad} review is organized using the methodology presented in \acs{fig} \ref{fig:wkfcad}. Methods embedded in the image regularization framework are presented before to focus on the image classification framework, the later being divided into \ac{cade} and \ac{cadx}. \Acl{tab} \ref{tab:sumpap} summarizes the forty-one different \ac{cad} studies reviewed in this paper. The first set of information reported is linked to the data acquisition such as the number of patients included in the study, the modalities acquired as well as the strength of the field of the scanner used. Subsequently, information about the prostate zones considered in the \ac{cad} analysis (\ac{pz} or \ac{cg}) are reported since that detecting \ac{cap} in the \ac{cg} is a more challenging problem and has received particular attention only in recent publications.

%%% Local Variables: 
%%% mode: latex
%%% TeX-master: "../g_lemaitre_state_of_the_art"
%%% End: 
