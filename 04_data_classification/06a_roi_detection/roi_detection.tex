\subsection{\ac{cade}: \acp{roi} detection/selection}\label{cade}

As discussed in the introduction and shown in \ac{fig}\,\ref{fig:wkfcad}, the image classification framework is often composed of a \ac{cade} and a \ac{cadx}. In this section, we will focus on studies embedding a \ac{cade} in their framework. Two approaches are considered to define a \ac{cade} (see \ac{tab}~\ref{tab:cade}): (i) voxel-based delineation and (ii) lesion segmentation.
The first strategy, which concerns the majority of the studies reviewed (see \ac{tab}~\ref{tab:cade}), is in fact linked to the nature of the classification framework. All voxels are considered as a possible lesion and the output of the framework will be pixels classified as lesion and non lesion.\robert{there are no works commented in the text on this first group? why?}
The second group of methods is composed of method implementing a lesion segmentation algorithm to delineate potential candidates to further obtain a diagnosis through the \ac{cadx}. This approach was borrowed from other application areas such as breast cancer. These methods are in fact very similar to the classification framework used in \ac{cadx} later.

Vos et al.~\cite{Vos2012} highlighted lesion candidates by detecting blobs in the \ac{adc} map. These candidates were filtered using some \textit{a priori} criteria such as \ac{si} or diameter. The candidate blobs detected are then filtered depending on their appearances (cf. maximum of the likelihood of the region, diameter of the lesion) and their \ac{si} in \ac{adc} and \ac{t2w} images. The detected regions are then used as inputs for the \ac{cadx}.

Litjens et al.~\cite{Litjens2011} used a pattern recognition approach in order to delineate the \acp{roi}. A blobness map was calculated in the same manner as previously in~\cite{Vos2010} using the multi-resolution Hessian blob detector on the \ac{adc} map, \ac{t2w} and pharmacokinetic parameters maps (see \ac{sec}\,\ref{subsec:featuredetection} for details about those parameters). Additionally, the position of the voxel $\mathbf{x}=\{x,y,z\}$ was used as a feature as well as the Euclidean distance of the voxel to the prostate center. Hence, the feature vectors were composed of eight features and a \ac{svm} classifier was trained using a \ac{rbf} kernel (see \ac{sec}\,\ref{subsec:classification} for more details).

Subsequently, Litjens et al. in~\cite{Litjens2012} modified this approach by including only features related to the blob detection on the different maps as well as the original \acp{si} of the parametric images. Two new maps were introduced based on texture. Instead of a \ac{svm} classifier, a \ac{knn} classifier was used. The candidate regions were then extracted by performing a local maxima detection followed by post-processing region-growing and morphological operations. 

In a similar way, Litjens et al. in~\cite{Litjens2014} used the same approach and added two new features: (i) a Gaussian texture bank on \ac{t2w} to create new maps and (ii) the \ac{dw} \ac{mri} image acquired with $b=800$ s.mm$^-2$. Three classifiers were tested: \ac{lda}, GentleBoost and Random forest. After evaluation, random forest was selected as classifier due to its overall performances.

%%% Local Variables: 
%%% mode: latex
%%% TeX-master: "../../g_lemaitre_state_of_the_art"
%%% End: 