\subsection{\ac{cadx}: Feature detection} \label{subsec:featuredetection}

Discriminative features which can be used to recognize \ac{cap} from healthy tissue have to be first detected. This processing is known in computer vision as feature extraction. However, feature extraction is also the name given in pattern recognition to some types of dimension reduction methods which will be presented in the next section. In order to avoid confusion between these two aspects, in this survey, the procedure ``detecting'' or ``extracting'' features from images and signals will be defined as feature detection. This section will summarize the different strategies employed for this task. The features used in the studies reviewed are summarized in \ac{tab}~\ref{tab:feat}.

\subsubsection{Image-based features}

This section will focus on image-based features detection. Two main strategies to detect features have been identified and used for the purpose of our classification: (i) voxel-wise detection and (ii) region-wise detection.

\setenumerate{listparindent=\parindent,itemsep=10px}
\setlist{noitemsep}
\begin{enumerate}[leftmargin=*]

\item[$-$] \textbf{\textit{Voxel-wise features:}} This strategy refers to the fact that a feature is extracted at each voxel location of an image. \ac{cap} as previously discussed (see \ac{tab}~\ref{tab:modmri}) can be discerned due to \ac{si} changes. Hence, intensity-based features are one of the most common features used to build the feature vector which has to be classified~\cite{Ampeliotis2007,Ampeliotis2008,Artan2009,Artan2010,Chan2003,Langer2009,Litjens2011,Litjens2012,Litjens2014,Liu2009,Niaf2011,Niaf2012,Viswanath2008a,Viswanath2011}. This type of feature consists simply of the \ac{si} of each voxel of the different \ac{mri} modalities.

  \Ac{si} changes can be viewed as heterogeneous regions and edge-based features are used in that regard. Each feature is computed by convolving the original image with an edge operator. Three of these operators are used in \ac{cad} systems: (i) Prewitt operator~\cite{Prewitt1970}, (ii) Sobel operator~\cite{Sobel1970} and (iii) Kirsch operator~\cite{Kirsch1971}. Results obtained with these operators vary, due to their different kernels. These features are commonly incorporated in the feature vector for further classification in the \ac{cad} systems reviewed~\cite{Niaf2011,Niaf2012,Tiwari2009a,Tiwari2010,Tiwari2013,Viswanath2008,Viswanath2011}.

  Gabor filters~\cite{Gabor1946,Daugman1985} offer another approach to extract information related to edges and texture and where integrated in three different \ac{cad} for \ac{cap}~\cite{Viswanath2008a,Viswanath2012,Tiwari2012}.

  Texture-based features provide other characteristics discerning \ac{cap} from healthy tissue. The most common texture analysis for image classification are co-occurrence matrices with their related statistics which were proposed in~\cite{Haralick1973} and are commonly used in \ac{cad} systems~\cite{Antic2013,Niaf2011,Niaf2012,Tiwari2009a,Tiwari2010,Tiwari2013,Viswanath2008,Viswanath2008a,Viswanath2011,Viswanath2012}.

  Fractal analysis and more precisely a local estimation of the fractal dimension~\cite{Benassi1998} describing the texture roughness at a specific location was used in~\cite{Lopes2011}. A wavelet-based method in a multi-resolution framework was used to estimate the fractal dimension. Cancerous tissue were characterized to have a higher fractal dimension than healthy tissue.

  Chan et al.~\cite{Chan2003} aimed to describe the texture using the frequency signature via the \acf{dct}~\cite{Ahmed1974} defining a neighbourhood of $7 \times 7$ pixels for each of the modalities that they used.

  In the same spirit, Viswanath et al. in~\cite{Viswanath2012} projected \ac{t2w} images into the wavelet space and used the coefficients obtained from the decomposition as features. The wavelet family used for the decomposition was the Haar wavelet.

  Litjens et al. in~\cite{Litjens2014} computed texture map based on \ac{t2w} images using a Gaussian filter bank~\cite{Leung2001}.

  The position of a voxel within the prostate was also considered a possible feature as in~\cite{Litjens2011,Litjens2014}. In these works, the Euclidean distance from each voxel to the prostate center as well as the individual distance in the three directions $x$, $y$ and $z$ are computed. Chan et al.~\cite{Chan2003} embedded the same information but this time using cylindrical coordinates $r$, $\theta$ and $z$ corresponding to the radius, azimuth and elevation respectively.

\item[$-$] \textbf{\textit{Region-wise features:}} Unlike the previous section, another strategy is to study an entire region and extract characteristic features corresponding to this region.

  The most common approach reviewed can be classified as statistical methods. First, a feature map are computed for the whole image and not any more for a single voxel. Then, \acp{roi} are defined and statistics are extracted from each of these regions. The first type of statistic is based on percentiles and is widely used~\cite{Antic2013,Litjens2011,Litjens2012,Litjens2014,Peng2013,Tiwari2009a,Tiwari2010,Tiwari2013,Viswanath2008,Viswanath2008a,Viswanath2011,Viswanath2012,Vos2008,Vos2008a,Vos2010,Vos2012}. In fact, once that a \ac{roi} is defined, the features corresponding to the $n^{\text{th}}$ percentile are used as feature. This $n$ can take any value between $0$ and $100$. This threshold is usually manually determined observing the distribution and corresponds to the best discriminant value differentiating malignant and healthy tissue. in addition, statistic-moments such as mean, standard deviation, kurtosis and skewness are also used~\cite{Ampeliotis2007,Ampeliotis2008,Antic2013,Niaf2011,Niaf2012,Peng2013}. Litjens et al. in~\cite{Litjens2014} also introduced a feature based on symmetry. They compute the mean of a candidate lesion as well as its mirrored counter-part and compute the quotient as feature.

  Another subset of features are anatomic which were also used in~\cite{Litjens2012,Litjens2014,Matulewicz2013}. Litjens et al. in~\cite{Litjens2012,Litjens2014} computed the volume, compactness and sphericity related to the region to integrate it in their feature vector to later classify. Matulewicz et al.~\cite{Matulewicz2013} introduced four features corresponding to the percentage of tissue belonging to the regions \ac{pz}, \ac{cg}, periurethral region or outside prostate region for the considered \ac{roi}.

  In contrast to anatomical are histogram-based feature descriptors. For instance, Liu et al.~\cite{Liu2013} introduced four different types of histogram-based features. The first type corresponds to the histogram of the \ac{si} of the image. The second type is the \acf{hog}~\cite{Dalal2005}. \Ac{hog} descriptor describes the local shape of the object of interest by using gradient directions distribution. The third histogram-based type used in~\cite{Liu2013} was shape context~\cite{Belongie2002}. The shape context is also a way to describe the shape of an object of interest. The last set of histogram-based feature extracted is based on the framework described in~\cite{Zhao2012} which is using the Fourier transform of the histogram created via \acf{lbp}~\cite{Ojala1996}.

  The last group of region-based feature is based on fractal analysis. The features proposed are based on estimating the fractal dimension which is a statistical index representing the complexity of what is analysed. Lv et al.~\cite{Lv2009} proposed two features based on fractal dimension: (i) texture fractal dimension and (ii) histogram fractal dimension. Lopes et al.~\cite{Lopes2011} proposed a 3D version to estimate the fractal dimension of a volume using wavelet decomposition.
\end{enumerate}

\subsubsection{\ac{dce}-based features}\label{subsubsec:fddce}

\ac{dce}-\ac{mri} is more commonly based on a \ac{si} analysis over time as presented in \ac{sec}\,\ref{subsubsec:mrimrsi}. The parameters extracted used in \ac{cad} system during the \ac{dce}-\ac{mri} analysis are presented.

\setenumerate{listparindent=\parindent,itemsep=10px}
\setlist{noitemsep}
\begin{enumerate}[leftmargin=*]

\item[$-$] \textbf{\textit{Whole-spectra approach:}} Some studies are using the whole \ac{dce} time series as feature vector~\cite{Ampeliotis2007,Ampeliotis2008,Tiwari2012,Viswanath2008a,Viswanath2008}. In some cases, the high-dimensional feature space is reduced using dimension reduction methods as it will be presented in the next section (see \ac{sec}\,\ref{subsec:featureselectionextraction}).

\item[$-$] \textbf{\textit{Semi-quantitative approach:}} Semi-quantitative approaches are based on mathematically modelling the \ac{dce} time series. The parameters modelling the signal are commonly used mainly due to the simplicity of their computation. Parameters included in semi-quantitative analysis are summarized in \ac{tab}~\ref{tab:semiqua} and also graphically depicted in \ac{fig}\,\ref{fig:dceparam}. A set of time features corresponding to specific amplitude level (start, maximum and end) are extracted. Then, derivative and integral features are also considered as discriminative and are commonly computed.

\item[$-$] \textbf{\textit{Quantitative approach:}} As presented in \ac{sec}\,\ref{subsubsec:mrimrsi}, quantitative approaches correspond to mathematical-pharmacokinetic models based on physiological exchanges. Four different models have been used in \ac{cad} for \ac{cap} systems. The most common model reviewed was the Brix model using three parameters $A$, $k_{ep}$ and $k_{el}$~\cite{Artan2009,Artan2010,Sung2011,Liu2009,Ozer2009,Ozer2010}. The Tofts model~\cite{Tofts1997} and more precisely the parameters $K_{trans}$, $k_{ep}$ and $v_e$ were used in~\cite{Langer2009,Litjens2011,Litjens2012,Litjens2014,Giannini2013,Niaf2011,Niaf2012,Mazzetti2011}.

  Mazzetti et al.~\cite{Mazzetti2011} and Giannini et al.~\cite{Giannini2013} used the Weibull function defined by two parameters. They also used another empirical model which is based on the West-like function and named the phenomenological universalities model~\cite{Castorina2006} using the three parameters the parameters $\beta$, $a_0$ and $r$.

  For all these models, the parameters are inferred using an optimization curve fitting approach.

\end{enumerate}

\subsubsection{\ac{mrsi}-based features}

\setenumerate{listparindent=\parindent,itemsep=10px}
\setlist{noitemsep}
\begin{enumerate}[leftmargin=*]

\item[$-$] \textbf{\textit{Whole spectra approach:}} As in the case of \ac{dce} analysis, one common approach is to incorporate the whole \ac{mrsi} spectra in the feature vector for classification~\cite{Kelm2007,Parfait2012,Tiwari2007,Tiwari2009,Tiwari2013,Tiwari2009a,Tiwari2010,Viswanath2008a,Matulewicz2013}. Sometimes post-processing involving dimension reduction methods is performed to reduce the complexity during the classification as it will be presented in \ac{sec}\,\ref{subsec:featureselectionextraction}.

\item[$-$] \textbf{\textit{Quantification approach:}} We can reiterate that in \ac{mrsi} only few biological markers (cf., choline, creatine and citrate metabolites mainly) are known to be useful to discriminate \ac{cap} and healthy tissue. Then, concentrations of these metabolites can be considered as a feature used for classification. In order to perform this quantification, four different approaches have been used. The QUEST~\cite{Ratiney2005}, AMARES~\cite{Vanhamme1997} and VARPRO~\cite{Coleman1993} models were used in~\cite{Kelm2007}. They are all time-domain quantification methods varying by the type of pre-knowledge embedded and the optimization approaches used to solve the quantification problem. Unlike the time-domain quantification approaches, Parfait et al.~\cite{Parfait2012} used the LcModel approach~\cite{Provencher1993} which solves the optimization problem in the frequency domain.

  Once the different concentrations are computed, Kelm et al.~\cite{Kelm2007} calculated the relative concentrations while Parfait et al.~\cite{Parfait2012} used each metabolite concentrations individually.

\item[$-$] \textbf{\textit{Wavelet decomposition approach:}} Tiwari et al.~\cite{Tiwari2012} performed a wavelet packet decomposition~\cite{Coifman1992} with the Haar wavelet basis function and used the coefficients of this decomposition as features for further classification.

\end{enumerate}

%%% Local Variables: 
%%% mode: latex
%%% TeX-master: "../../g_lemaitre_state_of_the_art"
%%% End: 