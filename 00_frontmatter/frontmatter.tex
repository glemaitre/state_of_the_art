\begin{frontmatter}

%% Title, authors and addresses

%% use the tnoteref command within \title for footnotes;
%% use the tnotetext command for the associated footnote;
%% use the fnref command within \author or \address for footnotes;
%% use the fntext command for the associated footnote;
%% use the corref command within \author for corresponding author footnotes;
%% use the cortext command for the associated footnote;
%% use the ead command for the email address,
%% and the form \ead[url] for the home page:
%%
%% \title{Title\tnoteref{label1}}
%% \tnotetext[label1]{}
%% \author{Name\corref{cor1}\fnref{label2}}
%% \ead{email address}
%% \ead[url]{home page}
%% \fntext[label2]{}
%% \cortext[cor1]{}
%% \address{Address\fnref{label3}}
%% \fntext[label3]{}

%\title{A survey of Computer-Aided Detection and Diagnosis systems for prostate cancer diagnosis based on mono and multi MRI/MRSI modalities}
\title{Computer-Aided Detection and Diagnosis for prostate cancer based on mono and multiparametric MRI/MRSI: A review}

%% use optional labels to link authors explicitly to addresses:
%% \author[label1,label2]{<author name>}
%% \address[label1]{<address>}
%% \address[label2]{<address>}

\author[label1,label3]{Guillaume~Lema\^itre\corref{cor1}}
\ead{guillaume.lemaitre@udg.edu}
\author[label3]{Robert~Mart\'i}
\ead{marly@eia.udg.edu}
\author[label3]{Jordi~Freixenet}
\ead{jordif@eia.udg.edu}
\author[label4]{Joan~C.~Vilanova}
\author[label2]{Paul~M.~Walker}
\ead{pwalker@u-bourgogne.fr}
\author[label1]{Fabrice~Meriaudeau}
\ead{fabrice.meriaudeau@u-bourgogne.fr}


\address[label1]{\scriptsize LE2I-UMR CNRS 6306, Universit\'{e} de Bourgogne, 12 rue de la Fonderie, 71200 Le Creusot, France}
\address[label2]{\scriptsize LE2I-UMR CNRS 6306, Universit\'{e} de Bourgogne, Avenue Alain Savary, 21000 Dijon, France}
\address[label3]{\scriptsize ViCOROB, Universitat de Girona, Campus Montilivi, Edifici P4, 17071 Girona, Spain}
\address[label4]{\scriptsize Girona Magnetic Resonance Center, 26 Carrer Joan Maragall, 17002 Girona, Spain}

\cortext[cor1]{Corresponding author.}

\begin{abstract}
Prostate cancer is reported to be the second most diagnosed cancer of men all over the world. In the last decades, new imaging techniques based on MRI have been developed improving the diagnosis task of radiologists. In practise, diagnosis can be affected by multiple factors reducing the chance to detect potential lesions. Computer-aided detection and computer-aided diagnosis have been designed to answer to these needs and provide help to radiologists in their daily duties. Research on computer-aided systems specifically focused for prostate cancer is a young technology and part of a dynamic field for the last ten years. This survey aimed to provide an overview of the researches carried out in this lapse of time and more precisely a comprehensive review of all the different stages composing the work-flow of computer-aided system. We also provide a comparison between these studies and potential avenues for future research are also discussed.
\end{abstract}

\begin{keyword}
%% keywords here, in the form: keyword \sep keyword
computer-aided detection \sep computer-aided diagnosis \sep CAD \sep magnetic resonance imaging \sep magnetic resonance spectroscopy imaging \sep computer vision 
%% MSC codes here, in the form: \MSC code \sep code
%% or \MSC[2008] code \sep code (2000 is the default)

\end{keyword}

\end{frontmatter}
